\documentclass[10pt,twoside]{article}

% Math symbols
\usepackage{amsmath}
\usepackage{amssymb}

% Headers/Footers
\usepackage{fancyhdr}

% Colors
\usepackage[usenames,dvipsnames]{color}

% Importing and manipulating graphics
\usepackage{graphicx}
\usepackage{subfig}
\usepackage{lscape}

% Misc packages
\usepackage{verbatim}
\usepackage{dcolumn}
\usepackage{ifpdf}
\usepackage{enumerate}

% PDF Bookmarks and hyperref stuff
\usepackage[
  bookmarks=true,
  bookmarksnumbered=true,
  colorlinks=true,
  filecolor=blue,
  linkcolor=blue,
  urlcolor=blue,
  hyperfootnotes=true
  citecolor=blue
]{hyperref}

% Improved hyperlinking to figures
% (include after hyperref)
\usepackage[all]{hypcap}

% Improved citation handling
% (include after the hyperref stuff)
\usepackage{cite}

% Pretty-print code
\usepackage{listings}

% -----------------------------------------------------------------
% Hyper-references
% -----------------------------------------------------------------
%
% This page has lots of good advice on hyper-references, including
% the use of the hypcap package and \phantomsection for generating
% labels to text.
%
% http://en.wikibooks.org/wiki/LaTeX/Labels_and_Cross-referencing
%
% -----------------------------------------------------------------
% Setup the margins
% -----------------------------------------------------------------
% Footer Template

% Set left margin - The default is 1 inch, so the following
% command sets a 1.25-inch left margin.
\setlength{\oddsidemargin}{0.25in}
\setlength{\evensidemargin}{0.25in}

% Set width of the text - What is left will be the right
% margin. In this case, right margin is
% 8.5in - 1.25in - 6in = 1.25in.
\setlength{\textwidth}{6in}

% Set top margin - The default is 1 inch, so the following
% command sets a 0.75-inch top margin.
%\setlength{\topmargin}{-0.25in}

% Set height of the header
\setlength{\headheight}{0.3in}

% Set vertical distance between the header and the text
\setlength{\headsep}{0.2in}

% Set height of the text
\setlength{\textheight}{8.5in}

% Set vertical distance between the text and the
% bottom of footer
\setlength{\footskip}{0.4in}

% -----------------------------------------------------------------
% Allow floats to take up more space on a page.
% -----------------------------------------------------------------

% see page 142 of the Companion for this stuff and the
% documentation for the fancyhdr package
\renewcommand{\textfraction}{0.05}
\renewcommand{\topfraction}{0.95}
\renewcommand{\bottomfraction}{0.95}
% dont make this too small
\renewcommand{\floatpagefraction}{0.35}
\setcounter{totalnumber}{5}

% -----------------------------------------------------------------
% Abbreviated symbols
% -----------------------------------------------------------------
\newcommand{\sinc}{\ensuremath{\,\text{sinc}}}
\newcommand{\rect}{\ensuremath{\,\text{rect}}}

% =================================================================
% The document starts here
% =================================================================
%
\begin{document}
\title{Altera JTAG-to-Avalon-MM Tutorial\\
{\large \em Version 1.1}}
\author{D. W. Hawkins (dwh@caltech.edu)}
%\date{\today}
% Date from v1.0 PDF
\date{March 14, 2012}
\maketitle

% No header/footer on the first page
\thispagestyle{empty}

\tableofcontents

% start the intro on an odd page
\cleardoublepage
%\clearpage

% Set up the header/footer
\pagestyle{fancy}
\lhead{Altera JTAG-to-Avalon-MM Tutorial}
\chead{}
\rhead{\today}
\lfoot{}
\cfoot{}
\rfoot{\thepage}
\renewcommand{\headrulewidth}{0.4pt}
\renewcommand{\footrulewidth}{0.4pt}

% Set the listings package language to Tcl
\lstset{language=Tcl}

% =================================================================
\section{Introduction}
% =================================================================

\begin{figure}[t]
  \begin{center}
    \includegraphics[width=0.95\textwidth]{figures/system_diagram.pdf}
  \end{center}
  \caption{System block diagram for host communication to an
  Altera Avalon hardware design.}
  \label{fig:system_diagram}
\end{figure}

A common question on the Altera forum is;
%
\begin{quote}
\em How can I use a USB-Blaster to communicate with my system design?
\end{quote}
%
The Altera USB-Blaster interface is used by the Quartus II
programmer to configure the FPGA, used by the SignalTap II logic 
analyzer for trace capture, and used by the NIOS II processor
tools for debugging and memory inspection, so it is reasonable
for users to assume that the interface can be used
to communicate with their designs. This tutorial demonstrates
how to implement this communications.

The target audience for this tutorial is developers new to Altera's 
SOPC Builder and Qsys system design tools.  This tutorial improves
the user experience with these tools, by providing a step-by-step
walk-though of the system design in Figure~\ref{fig:system_diagram}.
While many of the concepts covered will be foreign to the new user,
having an example of the end-to-end system design sequence, makes
reading and comprehension of the {\em extensive} tool and device 
documentation a little easier.

The tutorial shows how to create and simulate the hardware
design shown on the right of Figure~\ref{fig:system_diagram},
and then how to communicate with the design. The hardware design 
is based solely on Altera-provided IP components. The software
design uses the Altera Tcl-based tools {\em System Console} and
\verb+quartus_stp+ for the host-to-JTAG communications, and uses 
a (generic) Tcl GUI for the client application. Client-to-server
communications are performed using ASCII strings transported using
TCP/IP (sockets); the client can be easily replaced with one 
written in your favorite programming language.

The tutorial walks the reader through the creation, synthesis,
and simulation of SOPC and Qsys systems. Tcl scripts are provided
that automate the regeneration and simulation of the systems.
The tutorial points out some of the problems with the Altera-provided
IP and software; in part so that the reader can avoid the
problems, but also in the hope that Altera will rectify them.

% -----------------------------------------------------------------
\subsection*{Software Versions}
% -----------------------------------------------------------------

The tutorial was written using Altera Quartus 11.1sp1 and 
Modelsim-ASE ({\em Altera Starter Edition}) 10.0c. 
Appendix~\ref{sec:software_versions} provides details on
the operating systems tested, and differences with
earlier tool versions.

\textcolor{magenta}{\textbf{Update 5/21/2023:}} This tutorial
was uploaded to github~\cite{Hawkins_Altera_JTAG_to_Avalon_MM_Tutorial_2012}.
The repo was tagged as v1.0. The tutorial document was then
modified to add this note, to update the reference hyperlinks to
this document and the Altera JTAG protocol analysis, and then
the repo was tagged as v1.1. The github repo README 
includes instructions for synthesis with Intel Quartus Prime Lite 21.1.1. 
A new version of this tutorial using Intel Quartus Prime Platform 
Developer was created~\cite{Hawkins_Intel_JTAG_to_Avalon_MM_Tutorial_2023}.

\clearpage
% -----------------------------------------------------------------
\subsection*{Tutorial Source Code}
% -----------------------------------------------------------------

The tutorial zip file, 
\verb+altera_jtag_to_avalon_mm_tutorial.zip+~\cite{Hawkins_Altera_JTAG_to_Avalon_MM_Tutorial_2012}, 
unzips\footnote{See Appendix~\ref{app:tutorial_source} for
the recommended unzip methods under Windows and Linux.} to create
the directory layout shown in Table~\ref{tab:directory_layout}.
The path to the unzipped directory is referred to in this document
via the variable \verb+TUTORIAL+. For example, if you unzip the 
file into your Windows \verb+c:/temp+ directory, then paths in 
this document are referenced relative to
%
\begin{verbatim}
TUTORIAL = c:/temp/altera_jtag_to_avalon_mm_tutorial
\end{verbatim}
%
The example hardware design described in the text targets the
Arrow BeMicro-SDK Cyclone IV board, but the procedure works
equivalently well on any other development board. The example source 
contains completed designs for the Arrow BeMicro-SDK Cyclone IV, 
Arrow BeMicro Cyclone III, and Terasic DE2 Cyclone II boards.
The main difference between boards is in their top-level entities,
which contain all pins used on each board.
The Qsys or SOPC system is instantiated into the top-level entity.

% -----------------------------------------------------------------
% Tutorial Directory Layout
% -----------------------------------------------------------------
%
\begin{table}[t]
\caption{{\tt altera\_jtag\_to\_avalon\_mm\_tutorial} directory layout.}
\label{tab:directory_layout}
\begin{center}
\begin{tabular}{|l|l|}
\hline
Path & Description\\
\hline
&\\
\verb+doc/+ & Documentation\\
\verb+hdl/+ & HDL source code\\
\verb+tcl/+ & Tcl client/server source code\\
&\\
\hline
\multicolumn{2}{|l|}{\bf Simulation}\\
\hline
&\\
\verb+hdl/sopc_system/+ & SOPC system simulation\\
\verb+hdl/qsys_system/+ & Qsys system simulation\\
&\\
\hline
\multicolumn{2}{|l|}{\bf Synthesis}\\
\hline
&\\
\verb+hdl/boards/+      & Hardware targets\\
&\\
\verb+hdl/boards/bemicro_sdk/+             & Arrow BeMicro-SDK projects\\
\verb+hdl/boards/bemicro_sdk/sopc_system/+ & SOPC system synthesis\\
\verb+hdl/boards/bemicro_sdk/qsys_system/+ & Qsys system synthesis\\
\verb+hdl/boards/bemicro_sdk/share/+       & Board constraints\\
&\\
\verb+hdl/boards/bemicro/+             & Arrow BeMicro projects\\
\verb+hdl/boards/bemicro/sopc_system/+ & SOPC system synthesis\\
\verb+hdl/boards/bemicro/qsys_system/+ & Qsys system synthesis\\
\verb+hdl/boards/bemicro/share/+       & Board constraints\\
&\\
\verb+hdl/boards/de2/+             & Terasic DE2 projects\\
\verb+hdl/boards/de2/sopc_system/+ & SOPC system synthesis\\
\verb+hdl/boards/de2/qsys_system/+ & Qsys system synthesis\\
\verb+hdl/boards/de2/share/+       & Board constraints\\
&\\
\hline
\end{tabular}
\end{center}
\end{table}

\clearpage
% =================================================================
\section{SOPC Builder and Qsys}
% =================================================================

Altera provides two tools for graphically building hardware
systems; the classic tool {\em SOPC Builder}, and the new tool 
{\em Qsys}. The main difference between tools is the interconnect
fabric, and the support for hierarchical designs in 
Qsys~\cite{Altera_Qsys_NoC_Whitepaper_2011}. 

The {\em Avalon Interface Specification} consists of bus protocols
and an interconnect fabric defined by 
Altera~\cite{Altera_Interface_Specification_SOPC_2011,
Altera_Interface_Specification_Qsys_2011}. The bus protocols
consist of two variants; {\em Avalon Memory Mapped (Avalon-MM)}
and {\em Avalon Streaming (Avalon-ST)}. The Avalon-MM protocol
is used to create systems like that shown in Figure~\ref{fig:system_diagram},
where multiple masters connect to multiple slaves,
and the masters control the slaves by performing read
and write accesses to addresses defined by the
system {\em memory map}. The Avalon-ST protocol
is used in data streaming applications, such as signal processing,
where data sources pass data onto data sinks.

SOPC Builder is used to create systems containing Avalon-MM 
components and Avalon-ST components. The Avalon fabric is used
to connect Avalon-MM masters to Avalon-MM slaves. SOPC Builder
automates the systematic and tedious task of creating address
decoding, bus arbitration, and multiplexing logic between masters and slaves.
The connections between Avalon-ST sources and sinks are 
point-to-point, so no fabric is required.

Qsys takes a different approach to the implementation of the 
interconnect fabric; master and slave transactions are
converted to packets, and those packets are transported
through a {\em network-on-chip}. This approach abstracts the
bus interface protocol of the masters and slaves, allowing
different bus protocols to interface to the network.
This allows Altera to continue to use the Avalon-MM protocol,
and to add support for ARM defined bus protocols such as the
\href{http://en.wikipedia.org/wiki/Advanced_Microcontroller_Bus_Architecture}
{Advanced Microcontroller Bus Architecture (AMBA)}
Advanced eXtensible Interface (AXI).
For example, a Qsys system can be created with Avalon-MM
masters and AXI slaves, and the interconnect fabric
performs the required bus protocol translation.
To create this system using SOPC Builder, you would need
to create an Avalon-MM-to-AMBA bridge to {\em adapt}
the AMBA slave before connecting it to the Avalon-MM fabric.

The following sections define the system shown in 
Figure~\ref{fig:system_diagram} first using SOPC Builder
and then using Qsys. The two designs allow you to contrast
the two tools.

\clearpage
% =================================================================
\section{SOPC Builder Design Flow}
% =================================================================

In this section, the system in Figure~\ref{fig:system_diagram}
is created using SOPC Builder, synthesized using Quartus, and
simulated using Modelsim. The design is implemented using
the Quartus GUI. Once the design is complete, I show how to
create Tcl scripts that can regenerate the system from the
minimum number of design files.

% -----------------------------------------------------------------
\subsection{Project Creation}
% -----------------------------------------------------------------

Start Quartus 11.1sp1 and create a new project targeting the BeMicro-SDK board;
\begin{itemize}
\item File$\rightarrow$New Project Wizard
\item {\em Directory, Name, Top-level Entity [page 1 of 5]}
\begin{itemize}
\item Working directory name: \verb+c:\temp\altera_jtag_to_avalon_mm_tutorial\sopc\+
\item Project and top-level entity name: \verb+tutorial+
\item Click {\em Next}
\item Click {\em Yes} when prompted to create the working directory
\end{itemize}
%
\item {\em Add Files [page 2 of 5]}
\begin{itemize}
\item Click {\em Next}
\end{itemize}
%
\item {\em Family \& Device Settings [page 3 of 5]}
\begin{itemize}
\item Device Family: use the pull-down menu to select \verb+Cyclone IV E+
\item In the Available Devices spreadsheet, select \verb+EP4CE22F17C7+
\item Click {\em Next}
\end{itemize}
%
\item Click {\em Finish}, leaving all other settings at their defaults
\end{itemize}

\clearpage
% -----------------------------------------------------------------
\subsection{SOPC Builder Component}
% -----------------------------------------------------------------
%
% Screen-shot saved with Paint to PNG
\begin{figure}[t]
  \begin{center}
    \includegraphics[width=0.95\textwidth]{figures/sopc_system.png}
  \end{center}
  \caption{SOPC Builder {\tt sopc\_system} design.}
  \label{fig:sopc_system}
\end{figure}

Start SOPC Builder and create the system shown in Figure~\ref{fig:sopc_system};
%
\begin{itemize}
\item Tools$\rightarrow$SOPC Builder
\item In the {\em Create New System} GUI select \verb+Verilog+ as
the target language, and name it \verb+sopc_system+.

Selecting Verilog is a {\em requirement}, since in the next section
I show how to simulate the system using the free version of Modelsim
supplied by Altera, i.e., Modelsim-ASE.
The Altera Avalon and Avalon Verification IP components are
written in Verilog and SystemVerilog, and Modelsim-ASE only supports
single language simulation. The full version of Modelsim can be used
for mixed language designs.

\item Under {\em Clock Settings}, rename the clock to \verb+clk+
(the default clock frequency of 50MHz is correct for the BeMicro-SDK).
%
\item Add the SOPC System components from the {\em Component Library};
\begin{itemize}
%
\item Add the Avalon-MM BFM master;
\begin{itemize}
\item Verification$\rightarrow$Simulation$\rightarrow$Altera Avalon-MM Master BFM

For previous versions of Quartus use;\newline
Avalon Verification Suite$\rightarrow$Altera Avalon-MM Master BFM
\item Uncheck {\em Use the burstcount signal} and click finish.
\item Right click on the component and rename it \verb+bfm_master+.
\end{itemize}
%
\item Add the JTAG-to-Avalon-MM master;
\begin{itemize}
\item Bridges$\rightarrow$Memory Mapped$\rightarrow$JTAG to Avalon Master Bridge

For previous versions of Quartus use;\newline
Bridges and Adapters$\rightarrow$Memory Mapped$\rightarrow$JTAG to Avalon Master Bridge
\item Accept the defaults, and click finish.
\item Right click on the component and rename it \verb+jtag_master+.
\end{itemize}
%
\item Add the LED parallel I/O slave;
\begin{itemize}
\item Peripherals$\rightarrow$Microcontroller Peripherals$\rightarrow$PIO
(Parallel I/O).
\item Accept the defaults, and click finish.
\item Right click on the component and rename it \verb+led_pio+.
\item Connect the slave to the two masters; it will be assigned the base 
address \verb+0x00000000+.
\end{itemize}
%
\item Add the button parallel I/O slave;
\begin{itemize}
\item Peripherals$\rightarrow$Microcontroller Peripherals$\rightarrow$PIO
(Parallel I/O).
\item Change the direction to {\em input}, and click finish.
\item Right click on the component and rename it \verb+button_pio+.
\item Connect the slave to the two masters; and change its base address to
\verb+0x00000010+.
\end{itemize}
%
\item Add the on-chip memory slave;
\begin{itemize}
\item Memories and Memory Controllers$\rightarrow$On-chip$\rightarrow$On-chip
Memory (RAM or ROM).
\item Uncheck {\em Initialize memory content}, and click finish.
\item Right click on the component and rename it \verb+onchip_ram+.
\item Connect the slave to the two masters; and change its base address to
\verb+0x00001000+.
\end{itemize}
%
\end{itemize}
%
\item The SOPC System should now look like that in Figure~\ref{fig:sopc_system}.
The warnings shown in the SOPC builder GUI can be ignored.
%
\item Click {\em Generate} to generate the system.

SOPC Builder pops-up a {\em Save changes?} dialog window asking if you want 
to {\em Save changes to unnamed?} Click the {\em Save} button to bring
up the {\em Save} dialog window. Enter the system name, \verb+sopc_system.sopc+,
and click {\em Save}\footnote{See Appendix~\ref{app:altera_bugs}, 
Note 1.}\phantomsection\label{bug:1}.

\item When the message {\em Info: System generation was successful} appears, click 
the {\em Exit} button (click {\em Save} again when it prompts you).

\end{itemize}

So what did this just create? In the project directory, you will
see Verilog files for the SOPC system components;
\verb+bfm_master.v+, \verb+jtag_master.v+, \verb+led_pio.v+,
\verb+button_pio.v+, and \verb+onchip_ram.v+, and
a Verilog file for the top-level SOPC system,
\verb+sopc_system.v+. Open the files and look at the Verilog
code; the BFM master, JTAG master, and on-chip RAM files
simply instantiate components, while the LED and button PIO
components contain Verilog code. The SOPC system file is
the most complicated, it contains automatically generated
interconnect code (the main reason for using the system
generation tool). The SOPC system file is not particularly
readable, however, there are important sections of the file
that are discussed in Section~\ref{sec:sopc_system_simulation}.

The SOPC system description is implemented in the XML file
\verb+sopc_system.sopc+. All of the Verilog files generated
by SOPC Builder can be considered as {\em intermediate} files,
and they can be deleted and regenerated; much like you would
consider object files as intermediate files when compiling and
linking programs.
%
Go ahead and delete all the SOPC system generated Verilog files
and other related files with \verb+sopc_system+ in their names
({\em except} of course for \verb+sopc_system.sopc+), 
delete the directories \verb+jtag_master+ and \verb+sopc_system_sim+ 
too. Open the SOPC Builder GUI, and you will see the system 
unchanged. Click {\em Generate} and all the files you just deleted
will be regenerated.
%
The point of this last exercise was to show you that the
SOPC System component can be regenerated from the single
file \verb+sopc_system.sopc+, so that is the file to preserve
when creating a project archive or checking the project into
a version control system.

\clearpage
% -----------------------------------------------------------------
\subsection{Top-Level Design}
% -----------------------------------------------------------------
\label{sec:sopc_system_top_level}

\lstset{language=Verilog}
\begin{figure}
\begin{center}
\begin{minipage}{0.6\textwidth}
\begin{lstlisting}
module sopc_system (
    // 1) global signals:
    clk,
    reset_n,

    // the_button_pio
    in_port_to_the_button_pio,

    // the_led_pio
    out_port_from_the_led_pio
);
\end{lstlisting}
\end{minipage}
\end{center}
\caption{SOPC Builder {\tt sopc\_system} top-level Verilog module.}
\label{fig:sopc_system_verilog}
\end{figure}
%

Figure~\ref{fig:sopc_system_verilog} shows the Verilog module 
generated for the SOPC system component (buried inside the
\verb+sopc_system.v+ file). So how should this component be used?

SOPC Builder components can be used in one of two ways; the
component can be considered {\em the top-level component}, or 
the component can be considered just {\em one component} in a
top-level design, i.e.,  a component that you instantiate
in a top-level design. I encourage you to use the
latter interpretation. Here is my argument;
%
\begin{itemize}
%
\item FPGA pin assignments are essentially invariant once the 
device is placed-and-routed on a PCB. The design may contain
multiple general purpose I/Os (GPIO), whose properties
can be changed, but the wiring associated with GPIO bit 0
in the schematic will always route to the same pin on the FPGA, 
regardless of what you would like to name that signal within your
specific Verilog project.
%
\item Design constraints for I/O signals are generally applied
to the pin names of the signals. To avoid having to copy identical
constraints between projects, it is convenient to define a
top-level entity for a board with a fixed set of port names,
along with a constraints file with the nominal constraints.
Design-specific constraints can then either replace or augment
the nominal constraints.
%
\item Figure~\ref{fig:sopc_system_verilog} shows the port names
of the example SOPC system design. These port names depend on
the SOPC system master and slave names. Any changes to the SOPC
master and slave names results in port name changes.

If the top-level SOPC component is used as your top-level design
file, then the port names become the {\em pin names}.
Pin constraints must then be applied to pin names that are
being generated by SOPC Builder. This is a design maintenance 
headache.

By instantiating the SOPC component in a top-level component,
you gain the advantage of being able to rename the ports to
the standard pin names used for that particular board.

Consider the case where a SOPC component output or input port is
not used on a particular board. If the SOPC component is 
instantiated in a top-level design, then you can leave unused
outputs disconnected, and you can drive unused inputs to static values.
You cannot do this if you use the SOPC component as the top-level
design, as all ports become pins, and all pins need assignments
(otherwise Quartus will assign default values that could be
incorrect and result in damage to your board).

You also gain the appreciation that the SOPC system is a
reuseable component. For example, the SOPC system developed in
this tutorial is instantiated in top-level designs for the
BeMicro, BeMicro-SDK, and DE2 boards.

\end{itemize}

Because the SOPC system can be considered as a component, rather than
a top-level design, it can be placed in its own source directory, 
with a simulation testbench. A board design that instantiates the
SOPC system then adds the source path to its project, 
generates the Verilog source in a work directory, and uses that code
for synthesis. The example source provided with this tutorial
shows this approach.

\vskip5mm
{\bf Caveat}: An SOPC System {\em should} be able to be treated
as a component, however, the \verb+.sopc+ file contains a
reference to the FPGA device part number used when creating
the system (the part number is displayed on the SOPC Builder GUI).
The example code provided with this tutorial shows that a
single SOPC system file can be used with multiple boards
containing different devices. The SOPC System was designed
targeting the BeMicro-SDK Cyclone IV E device, but is reused
unchanged for the BeMicro Cyclone III and DE2 Cyclone II devices.
The synthesis script for each board first copies
\verb+sopc_system.sopc+ from the common area to the board-specific 
project work directory. The script needs to do this, as SOPC Builder
generates the output files in the same location as the SOPC file.
The synthesis script then requests the user to start the SOPC
Builder GUI and to {\em generate} the SOPC system files\footnote{I have
not figured out how to automate this from Tcl. The Tcl shell 
environment variables do not appear to be setup appropriately to
{\tt exec} the command-line tool {\tt sopc\_builder}.}.
The synthesis script request to generate the SOPC system occurs 
before the device constraint has been setup, so the SOPC Builder
GUI will always generate a warning that the device currently selected
for the project, the Cyclone IV GX, does not match that in
the SOPC file. The warning can be ignored.

% -----------------------------------------------------------------
\subsection{Synthesis}
% -----------------------------------------------------------------
\label{sec:sopc_system_synthesis}

Now that we have established why you should {\em not} use the SOPC 
system as the top-level entity, we will ignore that advice,
and do it anyway. In the Quartus GUI, under the Project Navigator
window (located on the top-left of the GUI), click on the 
{\em Files} tab, and you should see the file \verb+sopc_system.qip+
listed. If you do not,  add it using the 
{\em Project$\rightarrow$Add/Remove Files in Project} menu.
Right click on the file, and select {\em Set as Top-Level Entity},
then synthesize the design (press the play button on the GUI).

Figure~\ref{fig:sopc_system_hierarchy} shows the post-synthesis 
hierarchy display for the SOPC system design (when 
synthesized as the top-level design entity).
The SOPC design uses a total of 986 LCs, with the majority used
by the JTAG hub (\verb+sld_hub+, 99 LCs), and the JTAG master
({\tt jtag\_master}, 816 LCs).
The Qsys design uses a similar number of LCs.

Before moving on, look at the Quartus II message window
(bottom left of the GUI, with the {\em Processing} tab
selected). Scroll up to the top of the messages, and then
scroll down until you see the warning text (highlighted
in blue). These warnings are generated by the Avalon-MM
BFM component. This component is used in simulation only.
The warnings appear because the authors of the BFM components
have {\em not} used synthesis directives correctly.
The Verilog source for the Avalon-MM BFM should have
an Avalon-MM master interface for simulation, and another
for synthesis. For synthesis, the Avalon-MM master interface
signals should be driven to deasserted levels, allowing the
synthesis tool to eliminate the logic (without generating
warnings). However, as it is currently implemented, 
Quartus generates a large number of
warnings about missing drivers and dangling pins!\phantomsection\label{bug:2}

% -----------------------------------------------------------------
% SOPC System hierarchy
% -----------------------------------------------------------------
%
% In Quartus, detach the window from the main GUI window and
% scale it to the desired size. Then ctrl-alt-printscr
%
% Screen-shot saved with Paint to PNG
\begin{landscape}
\begin{figure}[p]
  \begin{center}
    \includegraphics[width=210mm]
    {figures/sopc_system_hierarchy.png}
  \end{center}
  \caption{Quartus II hierarchy display for the
  {\tt sopc\_system} design. The design uses 986 LCs, with
  the bulk being used by the JTAG hub ({\tt sld\_hub}, 99 LCs)
  and the JTAG-to-Avalon-MM master ({\tt jtag\_master}, 816 LCs).}
  \label{fig:sopc_system_hierarchy}
\end{figure}
\end{landscape}

% -----------------------------------------------------------------
\subsection{Simulation}
% -----------------------------------------------------------------
\label{sec:sopc_system_simulation}

The SOPC System design can be simulated using two possible methods;
use the Avalon-MM master BFM documented in the Verification IP
Guide, or live-on-the-edge and use the completely undocumented
Verilog tasks hidden deep within the JTAG-to-Avalon-MM master 
implementation. The following sections describe both methods.

% -----------------------------------------------------------------
\subsubsection{SOPC Builder {\tt test\_bench}}
% -----------------------------------------------------------------
\label{sec:sopc_system_test_bench}

The observant user would have noticed that SOPC Builder has a simulate
option; what does that do? Open up the SOPC Builder GUI, click
on the {\em System Generation} tab, and check the {\em Simulation.
Create project simulator files.} check-box. Check that Modelsim-ASE
is setup by using {\em Tools$\rightarrow$Options},
highlighting {\em HDL Simulator}, and for the simulator
{\em Mentor Graphic's Modelsim-Altera} set the {\em Application Path},
eg., \verb+c:\software\altera\11.1sp1\modelsim_ase\win32aloem+,
and click {\em Finish}. Now click the {\em Run Simulator} button.
You will get an error about a missing \verb+.mpf+ file (this is
the Modelsim project file); the {\em Run Simulator} button should
not really be highlighted until the \verb+.mpf+ file is
present in the project\phantomsection\label{bug:3}. To create the file, 
click the SOPC Builder {\em Generate} button, and then once 
generation completes, click the {\em Run Simulator} button. 

{\em Run Simulator} starts Modelsim-ASE, 
changes directory to \verb+$TUTORIAL/sopc/sopc_system_sim+ directory, and
loads the project file. The Modelsim command \verb+where+ can be
used to display the current directory and project.
The SOPC system simulation is controlled by Tcl procedures defined
in the script \verb+setup_sim.do+. Open the script and look at
the Tcl commands (they are far from easy to read); the Tcl commands
determine if the version of Modelsim is the Altera Edition or not, sets up a 
\verb+vsim+ (simulator) command to run on a component 
called \verb+test_bench+, and creates commands to build library 
components and the \verb+sopc_system.v+ file. 

At the Modelsim console type \verb+do setup_sim.do+ to
source the Tcl procedures defined in the
script. Figure~\ref{fig:sopc_system_setup_sim} shows the Modelsim
console output generated by the script. Type the command
\verb+s+ to build the components and load the simulation.
The simulation should load without errors, alas, Modelsim
fails with the message \verb+# Error loading design+
(along with a message that the design unit for the
\verb+altera_avalon_mm_master_bfm+ could not be found)\phantomsection\label{bug:4}.
The simulation files generated by Quartus 11.1sp1 do not
include the Avalon-MM BFM source files (earlier versions of
Quartus work fine). The problem can be rectified by editing
\verb+setup_sim.do+ to add the BFM source files inside the
alias for the \verb+s+ command, i.e., change the code to
%
\begin{verbatim}
alias s "vlib work;
_init_setup

vlog -sv [file join $env(QUARTUS_ROOTDIR)]/../ip/altera/sopc_builder_ip/
verification/lib/verbosity_pkg.sv
vlog -sv [file join $env(QUARTUS_ROOTDIR)]/../ip/altera/sopc_builder_ip/
verification/lib/avalon_mm_pkg.sv
vlog -sv [file join $env(QUARTUS_ROOTDIR)]/../ip/altera/sopc_builder_ip/
verification/altera_avalon_mm_master_bfm/altera_avalon_mm_master_bfm.sv
 
vlog +incdir+.. ../sopc_system.v;
\end{verbatim}
%
(where the start and end of this script segment already exist in the file, and
each \verb+vlog+ command is on one line, i.e., 3 new lines are added
to the script).
After editing \verb+setup_sim.do+, type \verb+do setup_sim.do+ at
the Modelsim prompt to read the modified script, and then
\verb+s+ to build the simulation files.
This time, the command completes without error (there is a
warning about \verb+onchip_ram+ not being initialized, but
that can be ignored).

% -----------------------------------------------------------------
% setup_sim.do commands
% -----------------------------------------------------------------
\begin{figure}[p]
\begin{center}
\begin{minipage}{0.8\textwidth}
\begin{verbatim}
# @@ setup_sim.do 
# @@ 
# @@ Defined aliases: 
# @@ 
# @@ s -- Load all design (HDL) files. 
# @@ re-vlog/re-vcom and re-vsim the design. 
# @@ 
# @@ s_cycloneiv -- For Modelsim SE, compile Cyclone IV models. 
# @@ (Ignored in Modelsim AE.) 
# @@ 
# @@ s_stratixiv -- For Modelsim SE, compile Stratix IV models. 
# @@ (Ignored in Modelsim AE.) 
# @@ 
# @@ s_stratixv -- For Modelsim SE, compile Stratix V models. 
# @@ (Ignored in Modelsim AE.) 
# @@ 
# @@ w -- Sets-up waveforms for this design 
# @@ Each SOPC-Builder component may have 
# @@ signals 'marked' for display during 
# @@ simulation. This command opens a wave- 
# @@ window containing all such signals. 
# @@ 
# @@ l -- Sets-up list waveforms for this design 
# @@ Each SOPC-Builder component may have 
# @@ signals 'marked' for listing during 
# @@ simulation. This command opens a list- 
# @@ window containing all such signals. 
# @@ 
# @@ h -- print this message 
# @@ 
\end{verbatim}
\end{minipage}
\end{center}
  \caption{SOPC system Modelsim setup script ({\tt setup\_sim.do}) Tcl commands.}
  \label{fig:sopc_system_setup_sim}
\end{figure}
% -----------------------------------------------------------------

\clearpage
Load the Modelsim wave window with the default SOPC system signals
by typing the \verb+w+ command; this populates the wave window with the
\verb+onchip_ram+ signals. Not very exiting is it?
Type \verb+add wave *+ to add the clock, reset, LED output, and button input
signals. Type \verb+run 1 us+ to run the simulation for 1$\mu$s.
In the Modelsim console, you will see a message output by the
Avalon-MM master BFM. Look at the wave window, and zoom to show
the simulation time from 0 to 1us (by clicking on the magnifying 
glass with the solid blue center). What happened in
the wave window? Well, if we had not added the clock and reset, 
a whole lot of nothing (well, 1$\mu$s of nothing really)!
By adding the reset and clock, at least we see some activity. 

This exercise shows that there is {\em no free lunch}; just because
SOPC Builder has a simulate button, does not mean it will write your
simulation testbench for you, all it does is provide the infrastructure for
{\em you} to write your simulation testbench.

To see what infrastructure Altera provides, open up the \verb+sopc_system.v+
file and scroll to the bottom of it. There are a couple of things to
observe;
%
\begin{itemize}
\item There is a module called \verb+test_bench+. This is the testbench
that you just simulated in Modelsim. The testbench contains a clock
generator, a reset generator, and the device under test; the SOPC system
component. Not much of a testbench really.
%
\item In the Verilog source, above the \verb+test_bench+ component,
Verilog \verb+include+ statements are used to inline a mixture of code
from the Quartus install directory, code copied to the project 
directory, and generated code. Appendix~\ref{app:altera_bugs} 
Notes 5\phantomsection\label{bug:5} and 6\phantomsection\label{bug:6a}
have comments on the disadvantages of using this technique to
resolve source code dependencies.

The Tcl simulation script described in Section~\ref{sec:sopc_system_tcl_scripts}
uses the \verb+vsim+ \verb|+incdir+| command line option to compile
the source that is \verb+include+'d inline in the Verilog source.
For this simulation to work, make sure to check the {\em Simulation}
check box under the {\em System Generation} tab in the SOPC Builder GUI.
\end{itemize}

So how then do we simulate this system? The Altera Verification IP
suite shows you one option; you create your own testbench and then
instantiate \verb+test_bench+ as the clock and reset generator.
Personally, I do not like that solution, as you lose control of
the reset line. Rather, I recommend ignoring \verb+test_bench+ 
entirely. The only useful simulation code that SOPC Builder 
generates are some of the script commands within \verb+setup_sim.do+;
the lines of code telling you the paths to the components included
in the project, and the arguments to the \verb+vsim+ command to
get the simulation to run (without generating lots of warnings).

\clearpage
% -----------------------------------------------------------------
\subsubsection{Avalon-MM Master BFM}
% -----------------------------------------------------------------

The source code for the Avalon-MM master BFM testbench is
located at;
%
\begin{verbatim}
$TUTORIAL/hdl/sopc_system/test/sopc_system_bfm_master_tb.sv
\end{verbatim}
%
This testbench uses the Altera Verification IP Suite to generate
Avalon-MM master transactions.

The Avalon-MM master BFM testbench can be simulated as follows;
%
\begin{itemize}
\item Start Modelsim.

This can be performed from the SOPC Builder GUI by clicking on
the {\em Run Simulator} button or you can run the simulator directly.

\item Set the tutorial path variable
%
\begin{verbatim}
Modelsim> set TUTORIAL c:/temp/altera_jtag_to_avalon_mm_tutorial
\end{verbatim}
%
\item If you did not start Modelsim from the SOPC Builder GUI,
change directory to the SOPC simulation directory and reset the
work library mapping
%
\begin{verbatim}
Modelsim> cd $TUTORIAL/sopc/sopc_system_sim
Modelsim> vmap work work
\end{verbatim}
%
The first argument to \verb+vmap+ is the library name, and
the second is the path to that library, i.e., the command
maps the \verb+work+ library to the \verb+work/+
directory in the current directory.
%
\item Compile the SOPC Builder source
%
\begin{verbatim}
Modelsim> do setup_sim.do
Modelsim> s
\end{verbatim}
%
This step will only succeed if the script has been edited per
the instructions in the previous section.
%
\item Compile the testbench
%
\begin{verbatim}
VSIM> vlog -sv $TUTORIAL/hdl/sopc_system/test/sopc_system_bfm_master_tb.sv
\end{verbatim}
%
\item Run the simulation 
%
\begin{verbatim}
VSIM> vsim -t ps +nowarnTFMPC sopc_system_bfm_master_tb
VSIM> do $TUTORIAL/hdl/sopc_system/scripts/sopc_system_bfm_master_tb.do
VSIM> run -a
\end{verbatim}
%
where the \verb+vsim+ \verb|+nowarnTFMPC| option suppresses warnings 
about missing connections (this argument was copied from \verb+setup_sim.do+),
and the \verb+.do+ file populates the wave window.
%
\item Modify the testbench source, recompile, and rerun the simulation via
%
\begin{verbatim}
VSIM> vlog -sv $TUTORIAL/hdl/sopc_system/test/sopc_system_bfm_master_tb.sv
VSIM> restart -f; run -a
\end{verbatim}
%
\end{itemize}

Figure~\ref{fig:sopc_system_bfm_master_tb} shows the Modelsim console
output produced by the testbench. The testbench checks the operation of the
LEDs, push buttons, and several locations in RAM. The testbench code
contains SystemVerilog assertions that would have generated an error
message for any failed test.


\begin{figure}
\begin{center}
\begin{minipage}{0.8\textwidth}
\begin{verbatim}
# ==============================================================
# JTAG-to-Avalon-MM SOPC System Testbench (using the BFM master)
# ==============================================================
#  * Deassert reset
# 
# --------------------------------------------------------------
# 1: Test the LEDs.
# --------------------------------------------------------------
#  * Write 0xAA to the LEDs
#    - LED register value = aah
#    - LED port value = aah
#  * Walking 1's test
#    - LED port value = 01h
#    - LED port value = 02h
#    - LED port value = 04h
#    - LED port value = 08h
#    - LED port value = 10h
#    - LED port value = 20h
#    - LED port value = 40h
#    - LED port value = 80h
# 
# --------------------------------------------------------------
# 2: Test the push buttons.
# --------------------------------------------------------------
#  * Push button value = 55h
#  * Walking 1's test
#    - Push button value = 01h
#    - Push button value = 02h
#    - Push button value = 04h
#    - Push button value = 08h
#    - Push button value = 10h
#    - Push button value = 20h
#    - Push button value = 40h
#    - Push button value = 80h
# 
# --------------------------------------------------------------
# 3: Test the on-chip RAM.
# --------------------------------------------------------------
#  * Fill 1024 locations of RAM with an incrementing count
#  * Read and check the RAM
# 
# ==============================================================
# Simulation complete.
# ==============================================================
\end{verbatim}
\end{minipage}
\end{center}
  \caption{Modelsim console output for the SOPC Builder
  Avalon-MM BFM master testbench,
  {\tt sopc\_system\_bfm\_master\_tb}.}
  \label{fig:sopc_system_bfm_master_tb}
\end{figure}

\clearpage
% -----------------------------------------------------------------
\subsubsection{JTAG-to-Avalon-MM Master}
% -----------------------------------------------------------------

Using only the Avalon-MM BFM master to test your design violates 
the principle
%
\begin{quote}
{\em ``Test what you fly, and fly what you test''}
\end{quote}
%
The main idea of the principle that you should {\em test what you 
use}. Using the Avalon-MM BFM to generate Avalon-MM transactions
{\em does not} test the JTAG-to-Avalon-MM master logic,
which is what generates the Avalon-MM master transactions in the
actual hardware. How then can you be sure that your hardware design
will function correctly? Arguably, you could test individual
components using the Avalon Verification IP suite, which would
then give you higher confidence that the components within the
system are functionally correct. The final SOPC system should still
have its own testbench, since the system has a {\em fabric} that is
specific to the design. So, lets use the Verification IP suite to test
the JTAG-to-Avalon Master $\dots$ Oh, but there is no documented way to
simulate the JTAG-to-Avalon-MM master interface, bummer.
We won't let the lack of documentation stop us though, continue 
reading!

The Altera wiki entry
\href{http://www.alterawiki.com/wiki/Avalon-ST_JTAG_Interface_PLI_Simulation_Mode}
{Avalon-ST\_JTAG\_Interface\_PLI\_Simulation\_Mode} shows
how to exercise the JTAG-to-Avalon-MM master in simulation using Verilog PLI
(Programming Language Interface). The PLI method uses a socket connection
between System Console and Modelsim, and {\em magically} (the code
is hidden in a Java library) generates transactions within the testbench.
While the PLI interface could be useful for some applications,
eg., developing a software interface, it is not appropriate for
use in self-verifying (automated) testbenches.
%
A self-verifying testbench should run completely within the Modelsim 
simulator, and the easiest way to do that, is to implement the testcase 
generator, the assertion logic, and the device under test using a
hardware description language.

The JTAG-to-Avalon-MM master consists of a JTAG-to-Avalon-ST interface
that converts JTAG transactions into byte streams in and out of the 
design, bytes-to-packets conversion logic that encodes and decodes
a binary protocol transported over the byte streams.
The binary protocol encodes whether to perform an Avalon-MM read or 
write transaction, and the response for each transaction type.
The packets to transactions component converts the commands
into Avalon-MM master commands.
The JTAG-to-Avalon-ST, bytes-to-packets, packets-to-transactions,
packets-to-bytes, and JTAG-to-Avalon-MM components are only
partially documented in the Altera literature. 
See~\cite{Hawkins_Altera_JTAG_to_Avalon_Analysis_2012} for
a detailed analysis of the JTAG-to-Avalon components.

The analysis document~\cite{Hawkins_Altera_JTAG_to_Avalon_Analysis_2012}
shows that buried deep within the source code for the JTAG-to-Avalon-ST
component is the logic for the JTAG node (which connects to the JTAG hub).
Hidden within the JTAG node are Verilog tasks for performing low-level
JTAG operations. Figure~\ref{fig:sopc_system_sld_node} shows the path 
to the JTAG node in the testbench developed for this section.
The JTAG node is highlighted in the figure, and the node's Verilog 
tasks are listed beneath it.
If you know how to generate JTAG transactions, and know how to 
use them to generate byte streams, then you have the makings of 
a JTAG-to-Avalon-MM master simulation. I won't bore you with the
details, the morbidly curious can read the testbench code and 
the analysis document.

The source code for the JTAG-to-Avalon-MM master testbench is
located at;
%
\begin{verbatim}
$TUTORIAL/hdl/sopc_system/test/sopc_system_jtag_master_tb.sv
\end{verbatim}
%
This testbench uses the undocumented Verilog tasks in the JTAG node
to generate byte-streams which encode Avalon-MM master transactions
for the packets-to-transactions component.

% -----------------------------------------------------------------
% JTAG-to-Avalon-MM bridge path to the SLD node
% -----------------------------------------------------------------
%
% Compile the sopc_system_jtag_master_tb.sv file.
%
% Detach the Modelsim window, scale to the desired size, highlight
% the JTAG node, capture the window using ctrl-alt-printsrc, and
% paste into Paint. Save as PNG.
%
\begin{figure}[p]
  \begin{center}
    \includegraphics[width=0.95\textwidth]{figures/sopc_altera_jtag_sld_node_hierarchy.png}
  \end{center}
  \caption{Altera JTAG node location in the SOPC Builder
  JTAG-to-Avalon-MM master testbench,
  {\tt sopc\_system\_jtag\_master\_tb}. The tasks under the
  highlighted JTAG {\tt node}, i.e., {\tt reset\_jtag\_state} down
  to {\tt shift\_one\_bit},
  are used to simulate the JTAG-to-Avalon-MM bridge.}
  \label{fig:sopc_system_sld_node}
\end{figure}
% -----------------------------------------------------------------

\clearpage
The JTAG-to-Avalon-MM master testbench can be simulated as follows;
%
\begin{itemize}
\item Start Modelsim.

This can be performed from the SOPC Builder GUI by clicking on
the {\em Run Simulator} button or you can run the simulator directly.

\item Set the tutorial path variable
%
\begin{verbatim}
Modelsim> set TUTORIAL c:/temp/altera_jtag_to_avalon_mm_tutorial
\end{verbatim}
%
\item If you did not start Modelsim from the SOPC Builder GUI,
change directory to the SOPC simulation directory and reset the
work library mapping
%
\begin{verbatim}
Modelsim> cd $TUTORIAL/sopc/sopc_system_sim
Modelsim> vmap work work
\end{verbatim}
%
The first argument to \verb+vmap+ is the library name, and
the second is the path to that library, i.e., the command
maps the \verb+work+ library to the \verb+work/+
directory in the current directory.
%
\item Compile the SOPC Builder source
%
\begin{verbatim}
Modelsim> do setup_sim.do
Modelsim> s
\end{verbatim}
%
This step will only succeed if the script has been edited per
the instructions in Section~\ref{sec:sopc_system_test_bench}.
%
\item Compile the testbench
%
\begin{verbatim}
VSIM> vlog -sv $TUTORIAL/hdl/sopc_system/test/sopc_system_jtag_master_tb.sv
\end{verbatim}
%
\item Run the simulation 
%
\begin{verbatim}
VSIM> vsim -t ps +nowarnTFMPC sopc_system_jtag_master_tb
VSIM> do $TUTORIAL/hdl/sopc_system/scripts/sopc_system_jtag_master_tb.do
VSIM> run -a
\end{verbatim}
%
where the \verb+vsim+ \verb|+nowarnTFMPC| option suppresses warnings 
about missing connections (this argument was copied from \verb+setup_sim.do+),
and the \verb+.do+ file populates the wave window.
%
\item Modify the testbench source, recompile, and rerun the simulation via
%
\begin{verbatim}
VSIM> vlog -sv $TUTORIAL/hdl/sopc_system/test/sopc_system_jtag_master_tb.sv
VSIM> restart -f; run -a
\end{verbatim}
%
\end{itemize}
%
The JTAG-to-Avalon-MM master testbench reproduces the test sequences 
performed by the Avalon-MM BFM master testbench shown in 
Figure~\ref{fig:sopc_system_bfm_master_tb}. The JTAG-to-Avalon-MM
master testbench console output is slightly different, as the
testbench first performs a JTAG protocol test, and then
duplicates the Avalon-MM BFM master testbench sequences.
The run-time of the JTAG testbench is much longer than that of the 
Avalon-MM BFM master testbench, due to the fact that the JTAG clock
is slower than the Avalon-MM clock, and byte stream transactions 
are serialized over JTAG. In practice, you should use the 
Avalon-MM BFM master for performing exhaustive tests on Avalon-MM 
slaves, and perform token tests with the JTAG simulation interface
to check that devices are connected correctly. Thus we finally 
reach our goal of {\em testing what we fly}.

\clearpage
% -----------------------------------------------------------------
\subsection{Synthesis and Simulation Scripts}
% -----------------------------------------------------------------
\label{sec:sopc_system_tcl_scripts}

Up until this point, you have been entering commands
in the Quartus or Modelsim console. What happens when you get bored
with typing, or forget the commands? Surely there is an easier way?
Yes, there is! Tcl scripts.

The BeMicro-SDK SOPC system project directory is located at;
%
\begin{verbatim}
$TUTORIAL/hdl/boards/bemicro_sdk/sopc_system/
\end{verbatim}
%
The project directory contains the synthesis script \verb+scripts/synth.tcl+,
and the top-level design file \verb+src/bemicro_sdk.sv+ (with the
SOPC system instantiated as a component).
The synthesis script can be run by starting Quartus, selecting
the Tcl console (if you cannot see it in the GUI, make it visible
using {\em View}$\rightarrow${\em Utility Windows}$\rightarrow${\em Tcl Console}),
changing to the project directory, and running the script, i.e.,
at the Quartus Tcl prompt
%
\begin{verbatim}
tcl> set TUTORIAL c:/temp/altera_jtag_to_avalon_mm_tutorial
tcl> cd $TUTORIAL/hdl/boards/bemicro_sdk/sopc_system/
tcl> source scripts/synth.tcl
\end{verbatim}
%
The script will determine that the SOPC System needs to be
generated. I have not figured out how to automate that from Tcl,
so the script asks you to run SOPC Builder, and {\em generate}
the SOPC system. Generate the system, and then exit the SOPC Builder GUI.
In the Tcl console, press the up-arrow to bring back the last 
command, and re-run the script. Quartus will then synthesize the
design. The top-level design connects 7-bits of the LED control
register to 7 of the 8 LEDs on the board, and blinks the other
LED at about 1Hz (so you can see the board is alive).
Thus implementing the design has been reduced to a few
Tcl commands and GUI button clicks---much less to remember!

The SOPC System simulations also have a Tcl script that will
setup the simulator without having to run the \verb+setup_sim.do+
script generated by Quartus.
%
The SOPC system simulation project directory is located at;
%
\begin{verbatim}
$TUTORIAL/hdl/sopc_system/
\end{verbatim}
%
That directory contains a \verb+scripts/+ directory containing
the simulation script \verb+sim.tcl+.
The simulation script can be run by starting Modelsim,
changing to the project directory, and running the script, i.e.,
%
\begin{verbatim}
ModelSim> set TUTORIAL c:/temp/altera_jtag_to_avalon_mm_tutorial
ModelSim> cd $TUTORIAL/hdl/sopc_system/
ModelSim> source scripts/sim.tcl
\end{verbatim}
%
The script will determine whether the SOPC System needs to be
generated (remember, Quartus generates code, so the simulation
needs to compile the generated code). The script is configured
to look for the SOPC system in the BeMicro-SDK project directory.
If it does not find it, it asks you to run Quartus to generate
it (which can be done using the \verb+synth.tcl+ script following
the sequence just described). Once the simulation script finds the Quartus
generated source, it compiles the testbenches, and creates two
Tcl procedures with the same names as the testbenches.
Either testbench can be run by typing the procedure name.
Each procedure issues the \verb+vsim+ command, populates the
wave window, and runs the simulation (see the script
for the procedure implementations). Again, much less typing!

The nice thing about the synthesis and simulation scripts is
that they show you the {\em minimum} number of source files needed
to reproduce both the Quartus and Modelsim projects. These are
the source files that you check into a code versioning system
(along with the synthesis and simulation scripts).

\clearpage
% =================================================================
\section{Qsys Design Flow}
% =================================================================

In the following sections, I show how to use the Qsys tool to
create a system, and then how to simulate it. Qsys is Altera's
the {\em new-and-improved} replacement for SOPC Builder. 
Reproducing the SOPC Builder design using Qsys helps contrast
the tools, and provides the reader with a reference design when
porting their own SOPC system designs to Qsys.

% -----------------------------------------------------------------
\subsection{Project Creation}
% -----------------------------------------------------------------

Start Quartus 11.1sp1 and create a new project targeting the BeMicro-SDK board;
\begin{itemize}
\item File$\rightarrow$New Project Wizard
\item {\em Directory, Name, Top-level Entity [page 1 of 5]}
\begin{itemize}
\item Working directory name: \verb+c:\temp\altera_jtag_to_avalon_mm_tutorial\qsys\+
\item Project and top-level entity name: \verb+tutorial+
\item Click {\em Next}
\item Click {\em Yes} when prompted to create the working directory
\end{itemize}
%
\item {\em Add Files [page 2 of 5]}
\begin{itemize}
\item Click {\em Next}
\end{itemize}
%
\item {\em Family \& Device Settings [page 3 of 5]}
\begin{itemize}
\item Device Family: use the pull-down menu to select \verb+Cyclone IV E+
\item In the Available Devices spreadsheet, select \verb+EP4CE22F17C7+
\item Click {\em Next}
\end{itemize}
%
\item Click {\em Finish}, leaving all other settings at their defaults
\end{itemize}

\clearpage
% -----------------------------------------------------------------
\subsection{Qsys Component}
% -----------------------------------------------------------------

% -----------------------------------------------------------------
% Qsys system screenshot
% -----------------------------------------------------------------
%
% Screen-shot saved with Paint to PNG
\begin{figure}[t]
  \begin{center}
    \includegraphics[width=0.95\textwidth]{figures/qsys_system.png}
  \end{center}
  \caption{Qsys {\tt qsys\_system} design.}
  \label{fig:qsys_system}
\end{figure}
% -----------------------------------------------------------------

Start Qsys and create the system shown in Figure~\ref{fig:qsys_system};
%
\begin{itemize}
\item Tools$\rightarrow$Qsys
\item The Qsys GUI starts up with a {\em Clock Source} already populated.
Right click, and rename it \verb+clk+.
%
\item Add the Qsys components from the {\em Component Library};
\begin{itemize}
%
\item Add the Avalon-MM BFM master;
\begin{itemize}
\item Verification$\rightarrow$Simulation$\rightarrow$Altera Avalon-MM Master BFM

For previous versions of Quartus use;\newline
Avalon Verification Suite$\rightarrow$Altera Avalon-MM Master BFM
\item Uncheck {\em Use the burstcount signal} and click finish.
\item Right click on the component and rename it \verb+bfm_master+.
\end{itemize}
%
\item Add the JTAG-to-Avalon-MM master;
\begin{itemize}
\item Bridges$\rightarrow$Memory Mapped$\rightarrow$JTAG to Avalon Master Bridge

For previous versions of Quartus use;\newline
Bridges and Adapters$\rightarrow$Memory Mapped$\rightarrow$JTAG to Avalon Master Bridge
\item Accept the defaults, and click finish.
\item Right click on the component and rename it \verb+jtag_master+.
\item Export the \verb+master_reset+ signal to the top-level, by clicking on the
{\em Click to export} text in the {\em Export} column,
{\em Reset Output} row, and enter the name \verb+resetrequest+.
This creates the top-level port \verb+resetrequest_reset+ on
the Qsys system (which can be viewed on the {\em HDL Example}
tab). This signal is an output from the JTAG master that is intended
for use as a JTAG controlled reset source (the hardware examples
instead use this signal to control an LED).
\end{itemize}
%
\item Add the LED parallel I/O slave;
\begin{itemize}
\item Peripherals$\rightarrow$Microcontroller Peripherals$\rightarrow$PIO
(Parallel I/O).
\item Accept the defaults, and click finish.
\item Right click on the component and rename it \verb+led_pio+.
\item Export the I/O to the top-level, by clicking on the
{\em Click to export} text in the {\em Export} column,
{\em Conduit Endpoint} row, and enter the name \verb+led+.
This creates the top-level port \verb+led_export+ on
the Qsys system (which can be viewed on the {\em HDL Example}
tab).
\item Connect the slave to the two masters; it will be assigned the base 
address \verb+0x00000000+.
\end{itemize}
%
\item Add the button parallel I/O slave;
\begin{itemize}
\item Peripherals$\rightarrow$Microcontroller Peripherals$\rightarrow$PIO
(Parallel I/O).
\item Change the direction to {\em input}, and click finish.
\item Right click on the component and rename it \verb+button_pio+.
\item Export the I/O to the top-level, by clicking on the
{\em Click to export} text in the {\em Export} column,
{\em Conduit Endpoint} row, and enter the name \verb+button+.
This creates the top-level port \verb+button_export+ on
the top-level Qsys system (which can be viewed on the {\em HDL Example}
tab).
\item Connect the slave to the two masters; and change its base address to
\verb+0x00000010+.
\end{itemize}
%
\item Add the on-chip memory slave;
\begin{itemize}
\item Memories and Memory Controllers$\rightarrow$On-chip$\rightarrow$On-chip
Memory (RAM or ROM).
\item Uncheck {\em Initialize memory content}, and click finish.
\item Right click on the component and rename it \verb+onchip_ram+.
\item Connect the slave to the two masters; and change its base address to
\verb+0x00001000+.
\end{itemize}
%
\end{itemize}
%
\item Connect the clock and reset signals;
\begin{itemize}
\item Connect the {\em Clock Source} component (\verb+clk+) {\em Clock Output}
(\verb+clk+)  to the clock input on each of the Avalon-MM masters and slaves.
\item Connect the {\em Clock Source} component (\verb+clk+ ) {\em Reset Output}
(\verb+clk_reset+) to the reset input on each of the Avalon-MM masters and slaves.
\end{itemize}
%
\item The Qsys system should now look like that in Figure~\ref{fig:qsys_system}.
The warnings shown in the Qsys GUI can be ignored.
%
\item Click on the {\em Generation} tab. Uncheck {\em Create block symbol
file (.bsf)}, then click the {\em Generate} button to generate the system.

Qsys generates a pop-up asking if you want to save changes to \verb+unnamed+.
Click {\em Save} and enter the system name; \verb+qsys_system.qsys+. 

\item When the message {\em Generate Complete. 0 Errors, 3 Warnings} appears, click 
the {\em Close} button, and then close the GUI using File$\rightarrow$Exit.

\end{itemize}
%
The Qsys system files are generated in a directory called \verb+qsys_system+.
This is an improvement over SOPC Builder, which would generate many of
the system files in the top-level of the Quartus work directory.
The top-level Qsys system module is located in the file
%
\begin{verbatim}
qsys_system/synthesis/qsys_system.v
\end{verbatim}
%
This file is an improvement over that generated by SOPC Builder in that it
contains only the \verb+sopc_system+ component module, making it easier to
comprehend. If you check the simulation option in the generate tab, 
and re-generate the system, another version of the file is created in
%
\begin{verbatim}
qsys_system/simulation/qsys_system.v
\end{verbatim}
%
and while the two files appear to be very similar, unfortunately, the code
generator generates the Verilog code in a different order, making it impossible
to compare the synthesis and simulation versions to determine actual code
differences. 
On a more positive note, Qsys does not use Verilog \verb+include+ 
statements, so that is another improvement over SOPC Builder
(see Appendix~\ref{app:altera_bugs} Note 5\phantomsection\label{bug:5}
for the SOPC Builder discussion).

The Qsys generated directories
%
\begin{verbatim}
qsys_system/synthesis/submodules
qsys_system/simulation/submodules
\end{verbatim}
%
contain {\em copies} of Quartus II installation library code 
(eg., the JTAG node, \verb+altera_jtag_sld_node.v+, which is used for 
JTAG-to-Avalon-MM master simulation). Appendix~\ref{app:altera_bugs} 
Note 6\phantomsection\label{bug:6b} discusses the disadvantages of 
copying what is essentially library source code into a project.

The Qsys system description is implemented in the XML file \verb+qsys_system.qsys+.
As with SOPC Builder, all of the generated files can be considered as
{\em intermediate} files; these files can be deleted and regenerated.
Go ahead and delete all of the Qsys files and directories, 
except for \verb+qsys_system.qsys+. Open the Qsys GUI, you will
be prompted for a system file, select \verb+qsys_system.qsys+,
and you will see the Qsys system unchanged. Select the {\em Generation}
tab, and then click the {\em Generate} button to regenerate the system.

\vskip5mm
{\bf Caveat}:\phantomsection\label{bug:7} 
When you select the {\em Generate} tab, you may notice that
the {\em Create block symbol file} check-box is checked again. The
settings selected on the {\em Generate} page are not preserved
in the \verb+.qsys+ file. See Appendix~\ref{app:altera_bugs}
Note 7 for a discussion.

\clearpage
% -----------------------------------------------------------------
\subsection{Top-Level Design}
% -----------------------------------------------------------------

Figure~\ref{fig:qsys_system_verilog} shows the top-level Qsys system
module port definitions from {\tt qsys\_system/synthesis/ qsys\_system.v};
the naming convention is a bit redundant isn't it?\phantomsection\label{bug:8} 
It seems kind of pointless that Altera allows you to define
a port name in the {\em Export} column in Figure~\ref{fig:qsys_system},
and then they go and munge the names to produce the port names in
Figure~\ref{fig:qsys_system_verilog}. From the comments after each port,
it appears that Altera's port naming convention is a mapping of a SystemVerilog
interface definition into a Verilog compatible port name.

% -----------------------------------------------------------------
% Qsys system Verilog module
% -----------------------------------------------------------------
%
\lstset{language=Verilog}
\begin{figure}[t]
\begin{center}
\begin{minipage}{0.9\textwidth}
\begin{lstlisting}
module qsys_system (
    output wire       resetrequest_reset, // resetrequest.reset
    output wire [7:0] led_export,         //          led.export
    input  wire [7:0] button_export,      //       button.export
    input  wire       reset_reset_n,      //        reset.reset_n
    input  wire       clk_clk             //          clk.clk
);
\end{lstlisting}
\end{minipage}
\end{center}
\caption{Qsys {\tt qsys\_system} top-level Verilog module.}
\label{fig:qsys_system_verilog}
\end{figure}
%

% -----------------------------------------------------------------
\subsection{Synthesis}
% -----------------------------------------------------------------

Section~\ref{sec:sopc_system_top_level} discusses the reasons why
an SOPC System or Qsys System should not be used as a top-level
component. Section~\ref{sec:sopc_system_synthesis} ignores that
advice to gauge the logic utilization of the system. 
We repeat that procedure here, to determine the Qsys system
logic utilization.

In the Quartus GUI, under the Project Navigator window 
(located on the top-left of the GUI), click on the 
{\em Files} tab, and then add the Qsys IP file
\verb+qsys_system.qip+ using the 
{\em Project$\rightarrow$Add/Remove Files in Project} menu.
Right click on the file, and select {\em Set as Top-Level Entity},
then synthesize the design (press the play button on the GUI).

Figure~\ref{fig:qsys_system_hierarchy} shows the post-synthesis 
hierarchy display for the Qsys system design (when 
synthesized as the top-level design entity).
The Qsys design uses a total of 1058 LCs, with the majority used
by the JTAG hub (\verb+sld_hub+, 99 LCs), and the JTAG master
({\tt jtag\_master}, 811 LCs). Figure~\ref{fig:sopc_system_hierarchy}
shows that the SOPC system design uses a similar number of LCs.

% -----------------------------------------------------------------
% Qsys System hierarchy
% -----------------------------------------------------------------
%
% In Quartus, detach the window from the main GUI window and
% scale it to the desired size. Then ctrl-alt-printscr
%
% Screen-shot saved with Paint to PNG
\begin{landscape}
\begin{figure}[p]
  \begin{center}
    \includegraphics[width=210mm]
    {figures/qsys_system_hierarchy.png}
  \end{center}
  \caption{Quartus II hierarchy display for the
  {\tt qsys\_system} design. The design uses 1058 LCs, with
  the bulk being used by the JTAG hub ({\tt sld\_hub}, 99 LCs)
  and the JTAG-to-Avalon-MM master ({\tt jtag\_master}, 811 LCs).}
  \label{fig:qsys_system_hierarchy}
\end{figure}
\end{landscape}

% -----------------------------------------------------------------
\subsection{Simulation}
% -----------------------------------------------------------------

The Qsys System design can be simulated using two possible methods;
use the Avalon-MM master BFM documented in the Verification IP
Guide, or live-on-the-edge and use the completely undocumented
Verilog tasks hidden deep within the JTAG-to-Avalon-MM master 
implementation. The following sections describe both methods.

% -----------------------------------------------------------------
\subsubsection{Qsys simulation configuration}
% -----------------------------------------------------------------

The Qsys {\em Generate} tab has several simulation options that
are described in the Quartus II Handbook, Volume 1, Chapter 5,
{\em Creating a System with Qsys} under {\em Simulating a Qsys
System}, on page 5-14~\cite{Altera_Quartus_Handbook_2011}.
This tutorial supplies the top-level testbench, so Qsys only
needs to generate the simulation model for the Qsys
system. The simulation model is created by setting the 
{\em Create simulation model} pull-down to {\em Verilog},
and then clicking the {\em Generate} button to generate
the system. The simulation model files are output in the 
project directory \verb+qsys_system/simulation+. 

The Qsys simulation option creates directories containing copies
of code from the Quartus installation, eg., the code in
the synthesis directory \verb+qsys_system/synthesis/submodules+,
is duplicated in the simulation directory
\verb+qsys_system/simulation/submodules+ 
(along with a few extra files copied from the Quartus 
install). Appendix~\ref{app:altera_bugs} 
Note 6\phantomsection\label{bug:6b} discusses the disadvantages of 
copying what is essentially library source code into a project.

% -----------------------------------------------------------------
% msim_setup.tcl commands
% -----------------------------------------------------------------
\begin{figure}[t]
\begin{center}
\begin{minipage}{\textwidth}
\begin{verbatim}
#
# List Of Command Line Aliases 
# 
# dev_com              -- Compile device library files 
# 
# com                  -- Compile the design files in correct order 
# 
# elab                 -- Elaborate top level design 
# 
# elab_debug           -- Elaborate the top level design with novopt option 
# 
# ld                   -- Compile all the design files and elaborate the top
#                          level design 
# 
# ld_debug             -- Compile all the design files and elaborate the top
#                        level design with -novopt 
# 
# List Of Variables 
# 
# TOP_LEVEL_NAME       -- Top level module name. 
# 
# SYSTEM_INSTANCE_NAME -- Instantiated system module name inside top level
#                         module. 
# 
# QSYS_SIMDIR          -- Qsys base simulation directory. 
#
\end{verbatim}
\end{minipage}
\end{center}
  \caption{Qsys system Modelsim setup script ({\tt msim\_setup.tcl}) Tcl commands.}
  \label{fig:qsys_system_msim_setup}
\end{figure}

% -----------------------------------------------------------------
\subsubsection{Avalon-MM Master BFM}
% -----------------------------------------------------------------

Simulation of the Qsys system in Modelsim is performed with
the assistance of the generated simulation script
\verb+qsys_system/simulation/mentor/msim_setup.tcl+.
The tutorial supplied testbench \verb+qsys_system_bfm_master_tb.sv+
can be simulated as follows;
%
\begin{itemize}
\item Start Modelsim
\item Change directory to the Mentor simulation directory
%
\begin{verbatim}
ModelSim> set TUTORIAL c:/temp/altera_jtag_to_avalon_mm_tutorial
ModelSim> cd $TUTORIAL/qsys/qsys_system/simulation/mentor
\end{verbatim}
%
\item Source the simulation script
%
\begin{verbatim}
ModelSim> source msim_setup.tcl
\end{verbatim}
%
Figure~\ref{fig:qsys_system_msim_setup} shows the script output.
%
\item Compile the device library source (not required for Modelsim-ASE)
%
%
\begin{verbatim}
ModelSim> dev_com
\end{verbatim}
%
\item Compile the Qsys source
%
\begin{verbatim}
ModelSim> com
\end{verbatim}
%
\item Compile the tutorial testbench
%
\begin{verbatim}
ModelSim> vlog -sv $TUTORIAL/hdl/qsys_system/test/qsys_system_bfm_master_tb.sv
         -L qsys_system_bfm_master
\end{verbatim}
%
The library path option, \verb+-L+, is required so that the SystemVerilog
verbosity package, compiled by the \verb+msim_setup.tcl+ script \verb+com+
procedure, is located.
%
\item Set the testbench to the top-level entity 
(\verb+TOP_LEVEL_NAME+ is used by the \verb+msim_setup.tcl+
script \verb+elab+ procedure)
%
\begin{verbatim}
ModelSim> set TOP_LEVEL_NAME qsys_system_bfm_master_tb
\end{verbatim}
%
\item Elaborate the testbench
%
\begin{verbatim}
ModelSim> elab +nowarnTFMPC
\end{verbatim}
%
where the argument to the command gets passed to \verb+vsim+ to
suppress warnings about missing connections.
%
\item Populate the wave window using
%
\begin{verbatim}
VSIM> do $TUTORIAL/hdl/qsys_system/scripts/qsys_system_bfm_master_tb.do
\end{verbatim}
%
\item Run the simulation
%
\begin{verbatim}
VSIM> run -a
\end{verbatim}
%
\end{itemize}
%
The testbench console output matches that generated by 
SOPC Builder in Figure~\ref{fig:sopc_system_bfm_master_tb}.

\clearpage
% -----------------------------------------------------------------
\subsubsection{JTAG-to-Avalon-MM Master}
% -----------------------------------------------------------------

The tutorial supplied testbench \verb+qsys_system_jtag_master_tb.sv+
can be simulated as follows;
%
\begin{itemize}
\item Start Modelsim
\item Change directory to the Mentor simulation directory
%
\begin{verbatim}
ModelSim> set TUTORIAL c:/temp/altera_jtag_to_avalon_mm_tutorial
ModelSim> cd $TUTORIAL/qsys/qsys_system/simulation/mentor
\end{verbatim}
%
\item Source the simulation script
%
\begin{verbatim}
ModelSim> source msim_setup.tcl
\end{verbatim}
%
Figure~\ref{fig:qsys_system_msim_setup} shows the script output.
%
\item Compile the device library source (not required for Modelsim-ASE)
%
%
\begin{verbatim}
ModelSim> dev_com
\end{verbatim}
%
\item Compile the Qsys source
%
\begin{verbatim}
ModelSim> com
\end{verbatim}
%
\item Compile the tutorial testbench
%
\begin{verbatim}
ModelSim> vlog -sv $TUTORIAL/hdl/qsys_system/test/qsys_system_jtag_master_tb.sv
         -L qsys_system_bfm_master
\end{verbatim}
%
The library path option, \verb+-L+, is required so that the SystemVerilog
verbosity package, compiled by the \verb+msim_setup.tcl+ script \verb+com+
procedure, is located.
%
\item Set the testbench to the top-level entity 
(\verb+TOP_LEVEL_NAME+ is used by the \verb+msim_setup.tcl+
script \verb+elab+ procedure)
%
\begin{verbatim}
ModelSim> set TOP_LEVEL_NAME qsys_system_jtag_master_tb
\end{verbatim}
%
\item Elaborate the testbench
%
\begin{verbatim}
ModelSim> elab +nowarnTFMPC
\end{verbatim}
%
where the argument to the command gets passed to \verb+vsim+ to
suppress warnings about missing connections.
%
\item Populate the wave window using
%
\begin{verbatim}
VSIM> do $TUTORIAL/hdl/qsys_system/scripts/qsys_system_jtag_master_tb.do
\end{verbatim}
%
\item Run the simulation
%
\begin{verbatim}
VSIM> run -a
\end{verbatim}
%
\end{itemize}
%
The testbench console output matches that generated by 
SOPC Builder in Figure~\ref{fig:sopc_system_bfm_master_tb}.

Figure~\ref{fig:qsys_system_sld_node} shows the path to the 
JTAG node within the JTAG-to-Avalon-MM bridge. The path to 
this node was first determined by elaborating \verb+qsys_system+
(rather than the testbench), and then using the Modelsim 
hierarchy window to determine the path to the node.

\clearpage
% -----------------------------------------------------------------
% JTAG-to-Avalon-MM bridge path to the SLD node
% -----------------------------------------------------------------
%
% Compile the qsys_system_jtag_master_tb.sv file.
%
% Detach the Modelsim window, scale to the desired size, highlight
% the JTAG node, capture the window using ctrl-alt-printsrc, and
% paste into Paint. Save as PNG.
%
\begin{figure}[p]
  \begin{center}
    \includegraphics[width=0.88\textwidth]{figures/qsys_altera_jtag_sld_node_hierarchy.png}
  \end{center}
  \caption{Altera JTAG node location in the Qsys system
  JTAG-to-Avalon-MM master testbench,
  {\tt qsys\_system\_jtag\_master\_tb}. The tasks under the
  highlighted JTAG {\tt node}, i.e., {\tt shift\_one\_byte} down
  to {\tt clear\_states},
  are used to simulate the JTAG-to-Avalon-MM bridge.}
  \label{fig:qsys_system_sld_node}
\end{figure}

\clearpage
% -----------------------------------------------------------------
\subsection{Synthesis and Simulation Scripts}
% -----------------------------------------------------------------

The Qsys design can be synthesized as follows;
%
\begin{itemize}
\item Start Quartus.
\item Change to the Qsys BeMicro-SDK project and
source the synthesis script
%
\begin{verbatim}
tcl> set TUTORIAL c:/temp/altera_jtag_to_avalon_mm_tutorial
tcl> cd $TUTORIAL/hdl/boards/bemicro_sdk/qsys_system
tcl> source scripts/synth.tcl
\end{verbatim}
%
\item The first time the script is run, it will copy the
\verb+qsys_system.qsys+ file to a work directory, and request
the user to manually run the Qsys GUI.

The \verb+.qsys+ file does not preserve the settings on the 
{\em Generate} tab, so uncheck {\em Create block symbol file}
and select {\em Verilog} for {Create Simulation Model}.
Click {\em Generate} to generate the system files.

Source the synthesis script (use the up-arrow in the Quartus
Tcl console to bring back the last command issued)
%
\begin{verbatim}
tcl> source scripts/synth.tcl
\end{verbatim}
%
\item Synthesis of the Qsys design should complete without error.

The Quartus {\em Processing} window will generate warning messages
(blue text). Warning messages relating to HDL coding style are
generated for some of the Altera IP, eg., missing connections for
the Avalon-MM BFM master (since it is for simulation only),
signals that were assigned and never read, and truncated signals.
Ideally, these warnings would be eliminated or suppressed by the
Altera IP developers, either by correcting the code or using synthesis
constraints.
\end{itemize}

The Qsys design can be simulated as follows;
%
\begin{itemize}
\item Start Quartus and {\em Generate} the Qsys system
(making sure the {\em Verilog} simulation option is checked).
\item Start Modelsim.
\item Change to the Qsys system directory and source the
simulation script
%
\begin{verbatim}
ModelSim> cd $TUTORIAL/hdl/qsys_system
ModelSim> source scripts/sim.tcl
\end{verbatim}
%
\item The simulation script uses the \verb+msim_setup.tcl+
script to compile the Qsys source, and creates two Tcl procedures
that can be used to run the simulation, i.e.,
%
\begin{verbatim}
# JTAG-to-Avalon-MM tutorial testbench procedures 
# ----------------------------------------------- 
#  
#   qsys_system_bfm_master_tb  - run the Avalon-MM BFM testbench 
#   qsys_system_jtag_master_tb - run the JTAG-to-Avalon-MM testbench 
\end{verbatim}
%
Issue either of these commands to run the respective simulation.

\item The simulation script creates a working directory called
\verb+mwork+, changes into that directory, sources and then calls
procedures in the \verb+msim_setup.tcl+ script. 

The \verb+msim_setup.tcl+ procedures copy files into the working
directory and create Modelsim library mappings in a subdirectory
called \verb+libraries+\phantomsection\label{bug:9}.
The library mappings are created with
{\em relative} path names, so the testbenches must be run from 
within the \verb+mwork+ directory, otherwise Modelsim cannot locate
the design components. 

To re-run the simulation script, quit the simulation (since Modelsim
will not allow you to change directories otherwise), change directory
to the top-level directory and source the script again, i.e.,
%
\begin{verbatim}
VSIM> quit -sim
ModelSim> cd $TUTORIAL/hdl/qsys_system
ModelSim> source scripts/sim.tcl
\end{verbatim}
%
Alternatively, if you simply edited the top-level testbench, recompile
it and restart the simulation via
%
\begin{verbatim}
VSIM> vlog -sv ../test/qsys_system_bfm_master_tb.sv -L qsys_system_bfm_master
VSIM> restart -f; run -a
\end{verbatim}
%
\end{itemize}

The tutorial source contains Qsys projects for Quartus synthesis on 
the BeMicro-SDK, BeMicro, and DE2 boards. The Qsys Modelsim simulation
script would ideally be board agnostic, however, it needs to use
Quartus generated Qsys source. By default the Qsys Modelsim 
simulation script checks for the existence of the BeMicro-SDK Quartus
work directory (which is setup by the synthesis script). If you 
synthesize the Qsys design for the BeMicro or DE2 board, and want
to simulate that source, you need to edit the 
the \verb+board+ variable in the \verb+sim.tcl+ script to
reflect the path to the board you are targeting.
However, since Quartus generates the same Qsys simulation files 
regardless of the board type, just follow the synthesis procedure 
above for the BeMicro-SDK board to create the files needed for 
Qsys simulation (no editing of tutorial scripts required).

\clearpage
% =================================================================
\section{Host-to-FPGA Communications}
% =================================================================
\label{sec:host_to_fpga_communications}

The goal of this tutorial is to demonstrate communications between
a client application and an Altera Avalon system such as that
shown in Figure~\ref{fig:system_diagram}. Previous sections
have shown how to construct the Altera Avalon hardware using
both SOPC Builder and Qsys tools. This section shows how to 
communicate with that hardware via the USB-Blaster interface
and Altera-provided applications.

% -----------------------------------------------------------------
\subsection{System Console}
% -----------------------------------------------------------------
%
The Quartus II Handbook, Volume 3, Chapter 10, {\em Analyzing and
Debugging Designs with the System Console}, describes the 
{\em System Console} interactive debugging console. 
Table 10-3 on pages 10-7 and 10-8 lists the console 
commands\cite{Altera_Quartus_Handbook_2011}.

The BeMicro-SDK board was configured with the Altera Avalon
system developed in this tutorial. 
Figure~\ref{fig:system_console_interactive} shows 
the System Console commands issued to interact with the 
BeMicro-SDK (the session was repeated for both the SOPC and
Qsys hardware configurations).

The System Console procedures were used to create higher-level
Tcl procedures for controlling the LEDs, reading the switches
and push-button, and for accessing SRAM. The script is located
in the BeMicro-SDK shared scripts directory,
%
\begin{verbatim}
$TUTORIAL/hdl/boards/bemicro_sdk/share/scripts/jtag_cmds_sc.tcl
\end{verbatim}
%
Similar scripts exist for the BeMicro and DE2 boards. Read the
scripts for the slight differences in procedures (due to the
slight difference in hardware available on these boards).
Figure~\ref{fig:system_console_jtag_cmds_sc} shows an
interactive System Console session with the BeMicro-SDK.
Note how the device
was never opened; the scripts use a Tcl global variable to track
whether the JTAG interface is open, and if it is not, the procedures
automatically open the JTAG interface. Read the script source
for details.

% -----------------------------------------------------------------
\subsection{\tt quartus\_stp}
% -----------------------------------------------------------------
%
The command-line tool \verb+quartus_stp+ can also be used for
JTAG access. Unfortunately, Altera does not provide Tcl procedures
for accessing the JTAG-to-Avalon-MM master component from within
\verb+quartus_stp+. The functionality of the JTAG-to-Avalon-MM 
master communications was reverse-engineered from the source 
code and SignalTap II logic analyzer traces 
in~\cite{Hawkins_Altera_JTAG_to_Avalon_Analysis_2012}, and
from that analysis a set of Tcl procedures was developed.
The tutorial source contains the procedures in the directory 
%
\begin{verbatim}
$TUTORIAL/tcl/altera_jtag_to_avalon_stp
\end{verbatim}
%
The procedures are written as a Tcl package.
Use of the package requires the user to either; configure the
environment variable \verb+TCLLIBPATH+ to point to the 
package directory, or the package directory can be copied into 
the Altera Tcl packages directory, eg.,
%
\begin{verbatim}
c:/software/altera/11.1sp1/quartus/common/tcl/packages
\end{verbatim}
%
Figure~\ref{fig:quartus_stp_interactive} shows an
interactive \verb+quartus_stp+ session with the BeMicro-SDK.
The session starts by loading the JTAG-to-Avalon-MM master
package, and then importing the Tcl procedure names
(saving you typing the namespace of the package as a prefix
to every command). For details on Tcl packages and
namespaces see~\cite{Welch_2000}.

% -----------------------------------------------------------------
% System Console Tcl session
% -----------------------------------------------------------------
%
% To get the text formatted in color, a minipage is used along with
% the color package. The font is changed to typewriter style, and
% the info text is changed to blue or green to match the appearance
% in system console.
%
\begin{figure}[p]
\hfil
\framebox{
\hspace{1mm}
\begin{minipage}{140mm}
\vspace{4mm}
\tt
\# Get the list of master services\\
\% \textcolor{blue}{set} masters [\textcolor{blue}{get\_service\_paths} master]\\
\{/devices/EP3C25|EP4CE22@1\#USB-0/(link)/JTAG/(110:132 v1 \#0)/phy\_0/master\}\\
\\
\# Select the first master\\
\% \textcolor{blue}{set} master [\textcolor{blue}{lindex} \textcolor{green}{\$masters} 0]\\
\\
\# Open the master service\\
\% \textcolor{blue}{open\_service} master \textcolor{green}{\$master}\\
\\
\# Write to the LEDs\\
\% \textcolor{blue}{master\_write\_32} \textcolor{green}{\$master} 0 0x55\\
\\
\# Read the push-button and switches SW[2:1]\\
\% \textcolor{blue}{master\_read\_32}  \textcolor{green}{\$master} 0x10 1\\
0x00000000\\
\\
\# Change SW[1] and re-read\\
\% \textcolor{blue}{master\_read\_32} \textcolor{green}{\$master} 0x10 1\\
0x00000001\\
\\
\# Change SW[2] and re-read\\
\% \textcolor{blue}{master\_read\_32} \textcolor{green}{\$master} 0x10 1\\
0x00000003\\
\\
\# Hold down the push-button and re-read\\
\% \textcolor{blue}{master\_read\_32} \textcolor{green}{\$master} 0x10 1\\
0x00000007\\
\\
\# Write four 32-bit locations in SRAM\\
\% \textcolor{blue}{master\_write\_32} \textcolor{green}{\$master} 0x1000 [\textcolor{blue}{list} 0x33221100 0x77665544 0xbbaa9988 0xffeeddcc]\\
\\
\# Read four 32-bit locations from SRAM\\
\% \textcolor{blue}{master\_read\_32}  \textcolor{green}{\$master} 0x1000 4\\
0x33221100 0x77665544 0xbbaa9988 0xffeeddcc\\
\\
\# Read sixteen 8-bit locations from SRAM\\
\% \textcolor{blue}{master\_read\_memory}  \textcolor{green}{\$master} 0x1000 16\\
0x00 0x11 0x22 0x33 0x44 0x55 0x66 0x77 0x88 0x99 0xaa 0xbb 0xcc 0xdd 0xee 0xff\\
\\
\# Close the master service\\
\% \textcolor{blue}{close\_service} master \textcolor{green}{\$master}\\
\vspace{1mm}
\end{minipage}
}\hfil
\caption{System Console interactive session controlling the 
BeMicro-SDK board configured with the Altera Avalon system
shown in Figure~\ref{fig:system_diagram}. The lines starting 
with {\tt \#} are comments and were not entered at the console.}
\label{fig:system_console_interactive}
\end{figure}

% -----------------------------------------------------------------
% System Console jtag_cmds_sc.tcl
% -----------------------------------------------------------------
%
\begin{figure}[p]
\hfil
\framebox{
\hspace{1mm}
\begin{minipage}{140mm}
\vspace{4mm}
\tt
\# Source the System Console JTAG Tcl procedures\\
\% set TUTORIAL c:/temp/altera\_jtag\_to\_avalon\_mm\_tutorial\\
\% source \$TUTORIAL/hdl/boards/bemicro\_sdk/share/scripts/jtag\_cmds\_sc.tcl\\
\\
\# Write/read the LEDs\\
\% led\_write 0x23\\
\% led\_read\\
0x23\\
\\
\# Read the switches\\
\% sw\\
2\\
\\
\# Read the push-button (not pressed)\\
\% pb\\
0\\
\\
\# Read the push-button (pressed)\\
\% pb\\
1\\
\\
\# Write/read the SRAM\\
\% sram\_write 0 0x12345678\\
\% sram\_read 0\\
0x12345678\\
\vspace{1mm}
\end{minipage}
}\hfil
\caption{System Console interactive session controlling the 
BeMicro-SDK board using the Tcl procedures implemented
in {\tt jtag\_cmds\_sc.tcl}. The lines starting 
with {\tt \#} are comments and were not entered at the console.}
\label{fig:system_console_jtag_cmds_sc}
\end{figure}

% -----------------------------------------------------------------
% quartus_stp interactive
% -----------------------------------------------------------------
%
\begin{figure}[p]
\hfil
\framebox{
\hspace{1mm}
\begin{minipage}{140mm}
\vspace{4mm}
\tt
\# Print TCLLIBPATH (needed to load the JTAG-to-Avalon-MM Tcl package)\\
tcl> puts \$env(TCLLIBPATH)\\
c:/temp/altera\_jtag\_to\_avalon\_mm\_tutorial/tcl/altera\_jtag\_to\_avalon\_stp\\
\\
\# Load the JTAG-to-Avalon-MM Tcl package and import the package commands
tcl> package require altera\_jtag\_to\_avalon\_stp\\
1.0\\
tcl> namespace import altera\_jtag\_to\_avalon\_stp::*\\
\\
\# Print the list of JTAG commands (commands prefixed with jtag)\\
tcl> info commands jtag*\\
jtag\_idcode jtag\_pulse\_nconfig jtag\_read jtag\_node\_id jtag\_node\_is\_bytestream jtag\_open jtag\_resetrequest jtag\_write jtag\_node\_is\_master jtag\_send jtag\_print\_hub\_info jtag\_close jtag\_usercode jtag\_print\_node\_info jtag\_number\_of\_nodes\\
\\
\# Open the JTAG interface\\
tcl> jtag\_open\\
JTAG: USB-Blaster [USB-0], FPGA: @1: EP3C25/EP4CE22 (0x020F30DD)\\
\\
\# Print the JTAG hub info\\
tcl> jtag\_print\_hub\_info\\
         Hub info: 0x8086E04\\
      VIR m-width: 4\\
      VIR n-width: 1\\
  Manufacturer ID: 0x6E\\
  Number of nodes: 1\\
       IP Version: 1\\
\\
\# Print the JTAG node info\\
tcl> jtag\_print\_node\_info\\
       Node index:    0\\
    Node instance:    0 (0x0)\\
Node manufacturer:  110 (0x6E)\\
          Node ID:  132 (0x84)\\
     Node purpose:    1 (0x1)\\
     Node version:    1 (0x1)\\
\\
\# Write/read the LEDs (the first argument is the JTAG node index)\\
\% jtag\_write 0 0 0x23\\
\% jtag\_read 0 0\\
0x23\\
\\
\# Write/read the SRAM\\
\% jtag\_write 0 0x1000 0x12345678\\
\% jtag\_read 0 0x1000\\
0x12345678\\
\vspace{1mm}
\end{minipage}
}\hfil
\caption{{\tt quartus\_stp} interactive session controlling the 
BeMicro-SDK board.}
\label{fig:quartus_stp_interactive}
\end{figure}

\clearpage
% -----------------------------------------------------------------
% quartus_stp client
% -----------------------------------------------------------------
%
% Screen-shot saved with Paint to PNG
\begin{figure}[t]
  \begin{center}
    \includegraphics[width=0.6\textwidth]{figures/jtag_client_gui.png}
  \end{center}
  \caption{{\tt quartus\_stp} JTAG client Tcl/Tk GUI.}
  \label{fig:quartus_stp_jtag_client}
\end{figure}

% -----------------------------------------------------------------
\subsection{Client/Server}
% -----------------------------------------------------------------
%
Altera does not provide a shared library or DLL for accessing
the JTAG interface from custom code, so how then do you write 
code in your favorite programming language, eg., C/C++, MATLAB,
or LabView?
One solution is to use System Console or \verb+quartus_stp+
to implement a TCP/IP server that provides hardware access, 
and implement your custom code as a TCP/IP client application.
The tutorial source contains Tcl-based server and client
applications in the directory
%
\begin{verbatim}
$TUTORIAL/tcl/jtag_client_server
\end{verbatim}
%
Figure~\ref{fig:quartus_stp_jtag_client} shows a Tcl/Tk based
client graphical user interface (GUI). The client can issue
read and write requests to the server, and the server then
performs those accesses on the hardware. The figure shows
the result of reading the first address in SRAM (which was
previously written with the value 0x12345678).

Figure~\ref{fig:quartus_stp_jtag_server} shows the Tcl/Tk based
server console output for \verb+quartus_stp+. Note the comment
{\tt Handle the client via a fileevent callback}. The Tcl
\verb+fileevent+ command is critical to the implementation
of a Tcl TCP/IP server~\cite{Welch_2000}.
Versions of System Console earlier than Quartus II version 11.1sp1
did not support the Tcl \verb+fileevent+ procedure, so they could
not implement a proper server, i.e., a server able to handle
multiple clients. Figure~\ref{fig:system_console_jtag_server}
shows the server output for System Console
(Quartus II version 11.1sp1).

The server application has a debug mode. If the server is started
via
%
\begin{verbatim}
% set debug 1
% source jtag_server.tcl
\end{verbatim}
%
then the server starts in a debug mode where hardware accesses
are not performed, and client read/write accesses are
performed on a server variable, eg., the client can write
to an address and then read it back. The server debug mode
allows you to test the client/server interface without having
access to hardware. It also allows the client/server TCP/IP 
communications path to be tested without hardware interaction.

To test the multiple client handling of the server (in either
hardware or debug mode), use the client to write the hexadecimal values
(address,data) =  (1000, 11111111), (1004, 22222222), (1008, 33333333),
and (100C, 44444444). Click on the client disconnect button,
then reconnect and read from the four hexadecimal addresses
1000, 1004, 1008, 100C. Start multiple \verb+quartus_stp+ clients, or
start the client in another Tcl/Tk tool, eg., the 
\href{http://www.activestate.com/activetcl}
{ActiveState ActiveTcl} Tcl/Tk shell \verb+wish84+, and use the
multiple clients to write and read to different server addresses. 

The client/server source also contains a command-line C language
client, \verb+jtag_client.c+. The client can be built at the
NIOS IDE shell using
%
\begin{verbatim}
bash-3.1$ gcc -Wall -o jtag_client jtag_client.c
\end{verbatim}
%
The client can write to multiple SRAM locations using
%
\begin{verbatim}
./jtag_client -w 0x1000 -d 0x11111111
./jtag_client -w 0x1004 -d 0x22222222
./jtag_client -w 0x1008 -d 0x33333333
./jtag_client -w 0x100C -d 0x44444444
\end{verbatim}
%
and read back using
%
\begin{verbatim}
./jtag_client -r 0x1000
./jtag_client -r 0x1004
./jtag_client -r 0x1008
./jtag_client -r 0x100C
\end{verbatim}
%
The NIOS IDE shell generates output showing the address and data.
The server console shows how the client opens and closes a
socket connection each time the client application runs.

The client/server example performs TCP/IP communications using
ASCII strings, and the server performs hardware access using
32-bit read/write commands. The performance bottleneck with
this approach is not the use of ASCII over TCP/IP, but the
server hardware access. When performing accesses to large
numbers of bytes, the JTAG 32-bit read/write commands
do not make the most efficient use of the underlying
JTAG bytestreams. If your custom server needs a hardware
access performance boost, then use the System Console 
\verb+master_read_memory+ and \verb+master_write_memory+ commands.
The JTAG bytestream performance is analyzed 
in~\cite{Hawkins_Altera_JTAG_to_Avalon_Analysis_2012}.
For simple hardware accesses, eg., updating LEDs, reading
switches and push buttons, and accessing sensors, the
32-bit read/write routines used in the client/server example
are more than sufficient.

% -----------------------------------------------------------------
% quartus_stp server
% -----------------------------------------------------------------
%
\begin{figure}
\hfil
\framebox{
\hspace{1mm}
\begin{minipage}{140mm}
\vspace{4mm}
\tt
tcl> source jtag\_server.tcl
\\
JTAG server running under quartus\_stp\\
\\
Open JTAG to access the JTAG-to-Avalon-MM master\\
JTAG: USB-Blaster [USB-0], FPGA: @1: EP3C25/EP4CE22 (0x020F30DD)\\
\\
Start the server on port 2540\\
\\
Wait for clients\\
\\
Accept sock1284 from 127.0.0.1 port 1427\\
Handle the client via a fileevent callback\\
SERVER (sock1284): jtag\_read 0x00000000\\
SERVER: jtag\_read 0x00000000\\
SERVER (sock1284): 0x00000045\\
SERVER (sock1284): jtag\_read 0x00001000\\
SERVER: jtag\_read 0x00001000\\
SERVER (sock1284): 0x12345678\\
\vspace{1mm}
\end{minipage}
}\hfil
\caption{{\tt quartus\_stp} JTAG server console output.}
\label{fig:quartus_stp_jtag_server}
%\end{figure}
%
% -----------------------------------------------------------------
% System Console (Quartus 11.1sp1) server
% -----------------------------------------------------------------
%
\vskip10mm
%\begin{figure}
\hfil
\framebox{
\hspace{1mm}
\begin{minipage}{140mm}
\vspace{4mm}
\tt
\% \textcolor{blue}{source} jtag\_server.tcl
\\
JTAG server running under system console\\
\\
This version of SystemConsole (11.1sp1 216) supports fileevent.\\
The server can support multiple clients.\\
\\
Open JTAG to access the JTAG-to-Avalon-MM master\\
\\
Start the server on port 2540\\
\\
Wait for clients\\
\\
Accept sock2024 from 127.0.0.1 port 1466\\
Handle the client via a fileevent callback\\
SERVER (sock2024): jtag\_read 0x00000000\\
SERVER: jtag\_read 0x00000000\\
SERVER (sock2024): 0x00000045\\
SERVER (sock2024): jtag\_read 0x00001000\\
SERVER: jtag\_read 0x00001000\\
SERVER (sock2024): 0x12345678\\
\vspace{1mm}
\end{minipage}
}\hfil
\caption{System Console JTAG server console output.}
\label{fig:system_console_jtag_server}
\end{figure}

\clearpage
\appendix
% =================================================================
\section{Software Versions}
% =================================================================
\label{sec:software_versions}

% -----------------------------------------------------------------
\subsection*{Current version of Quartus}
% -----------------------------------------------------------------

This tutorial was tested using the following Quartus II 11.1sp1
editions, and operating systems;
%
\begin{itemize}
\item The tutorial was written to be implemented using Quartus II 11.1sp1.
%
\item The tutorial was developed using Quartus II 11.1sp1
(full edition) under Windows XP Professional 32-bit. 
The synthesis scripts for all boards and both SOPC and Qsys systems
were fully tested. 
The simulation script was tested targeting the hardware generated
for the BeMicro-SDK. Each board was configured with the SOPC or
Qsys designs, and system console was used to interactively check 
each design.
%
\item Linux Centos 6.2 32-bit running Quartus II 11.1sp1 full-edition;
tested the synthesis scripts, the simulation script with the
BeMicro-SDK generated files, and hardware tested using system console.
%
%
\item Windows 7 Professional 64-bit running Quartus II 11.1sp1 full-edition
(32-bit and 64-bit);
tested the BeMicro-SDK synthesis and simulation scripts,
and hardware tested using system console.
%
\item Windows 7 Professional 64-bit running Modelsim SE 10.0c 64-bit;
tested the simulation scripts with the BeMicro-SDK generated files.

The Qsys BFM testbench fails to load due to the DLL
\verb+bytestream_pli.dll+ being incompatible with 64-bit Modelsim.
This DLL is not used by the testbench, so the BeMicro-SDK Qsys
project \verb+msim_setup.tcl+ script \verb+elab+ procedure was 
edited to remove the \verb+vsim+ PLI argument. 
The Qsys BFM testbench then loads and runs correctly.
%
\item Quartus II v11.1sp1 Web Edition for;
\begin{itemize}
\item Windows XP (32-bit)
\item Linux Centos 6.2 (32-bit)
\item Linux Ubuntu 11.10 (32-bit)
\end{itemize}
%
Each installation was used to test the BeMicro-SDK synthesis and 
simulation scripts, and hardware tested using system console.

Each Web Edition installation was tested running in a VirtualBox 
(version 4.1.8) virtual machine (VM). The VMs were tested 
from hosts running Windows XP, Centos 6.2, and Windows 7.
The USB-Blaster can be captured by the VM under Windows XP
and Centos 6.2, but not under Windows 7 (so the issue is
likely with the host VM support, rather than with the client).

The tutorial scripts use the environment variable \verb+QUARTUS_ROOTDIR+.
This variable is created automatically by Quartus II under 
Windows, but under Linux it needs to be defined by the user.
For example, under Linux, after installing Quartus II Web Edition
and Modelsim-ASE into the directory
\verb+/opt/altera/11.sp1_free+, the user \verb+.bashrc+ 
should be edited to add
%
\begin{verbatim}
export QUARTUS_ROOTDIR=/opt/altera/11.sp1_free/quartus
export PATH=$PATH:$QUARTUS_ROOTDIR/bin
export PATH=$PATH:/opt/altera/11.sp1_free/modelsim_ase/linuxaloem
\end{verbatim}
%
The commands \verb+quartus+ and \verb+vsim+ can then be used
from the bash shell.

\end{itemize}

\clearpage
% -----------------------------------------------------------------
\subsection*{Older versions of Quartus}
% -----------------------------------------------------------------

\begin{itemize}
\item {\bf SOPC Builder}

The SOPC Builder GUI and generated components have changed slightly between
Quartus version 10.1 and 11.1sp1. The slight differences in SOPC Builder
GUI paths to components is noted in the tutorial. The source
generated by SOPC Builder has changed with Quartus 11.1sp1,
eg., the JTAG-to-Avalon-MM bridge is now described solely by
a \verb+_hw.tcl+ file, with the component being dynamically
created by SOPC Builder. Previous versions of Quartus had an
explicit JTAG master component (which looked like it was SOPC
Builder generated source that was copied to the Quartus IP directory).

\item {\bf Qsys}

The Qsys GUI, generated scripts, and source have changed significantly
since 10.1, where Qsys was in beta format. No attempt was made to try
and support earlier versions of Qsys (it looked too painful).
\end{itemize}

% =================================================================
\section{Tutorial Source}
% =================================================================
\label{app:tutorial_source}

The tutorial zip file, 
\verb+altera_jtag_to_avalon_mm_tutorial.zip+, unzips to create
the directory layout shown in Table~\ref{tab:directory_layout}.
There are slight differences in how zip extraction tools work
under Windows and the bash shell (Cygwin or Linux).
To unzip the tutorial zip file into a directory named
\verb+altera_jtag_to_avalon_mm_tutorial+, perform the following;
%
\begin{itemize}
\item {\bf Windows extraction}

   Under Windows Explorer, select the zip file, and then
   right click and select {\em Extract all $\dots$}
  
\item {\bf Bash command line extraction}

Unzip using
%
\begin{verbatim}
unzip -d altera_jtag_to_avalon_mm_tutorial altera_jtag_to_avalon_mm_tutorial.zip
\end{verbatim}
%
If the \verb+-d+ option is not used, then the zip file creates
the directories \verb+doc+, \verb+hdl+, and \verb+tcl+ in the
current directory. Alternatively, you can first create
the directory and then unzip, eg.,
%
\begin{verbatim}
mkdir altera_jtag_to_avalon_mm_tutorial
cp altera_jtag_to_avalon_mm_tutorial.zip altera_jtag_to_avalon_mm_tutorial/
cd altera_jtag_to_avalon_mm_tutorial/
unzip altera_jtag_to_avalon_mm_tutorial.zip
\end{verbatim}
%
Using the \verb+-d+ option is recommended (as it involves less typing).
 
The contents of the zip file can be listed using
%
\begin{verbatim}
unzip -l altera_jtag_to_avalon_mm_tutorial.zip
\end{verbatim}
%
\end{itemize}

\clearpage
% =================================================================
\section{Altera Tool Improvement Recommendations}
% =================================================================
\label{app:altera_bugs}

During the development of the tutorial, problems experienced with
the user interface, the tool IP, or the tool design philosophy
were cross-referenced to this appendix. The text references the
following note numbers;
%
\begin{enumerate}
%
\item Page~\pageref{bug:1}: Why does SOPC Builder forget the SOPC System name?

When the {\em Generate} button is pressed to generate the SOPC System,
a pop-up dialog asks if the \verb+unnamed+ system should be saved.
The system was however named when the SOPC Builder GUI was started,
so this dialog should not be necessary.

\item Page~\pageref{bug:2}: Synthesis issues with the Avalon-MM Master BFM.

The Avalon-MM Master BFM component is intended for {\em simulation}-only.
However, it can be incorporated into a system that is both simulated and
synthesized. The source code for the component should include synthesis
directives that disable the component during synthesis (by tying signals
to deasserted levels), so that the synthesis tool can eliminate the
logic without generated warnings about missing drivers and dangling pins.

\item Page~\pageref{bug:3}: SOPC Builder {\em Run Simulator} Button.

The button should not be activated until the \verb+.mpf+ file has been
generated.

\item Page~\pageref{bug:4}: Missing source files for Avalon-MM Master BFM simulation.

The \verb+sopc_system.v+ verilog file does not include the packages required
to simulate the Avalon-MM Master BFM.

\item Page~\pageref{bug:5}: 
Using Verilog \verb+include+ statements to resolve
source dependencies.

In the SOPC Builder example, 
%
\begin{itemize}
\item In the \verb+sopc_system.v+ Verilog source, above the
\verb+test_bench+ component, Verilog \verb+include+ statements have
been used to include a mixture of code from the Quartus install directory,
code copied to the project directory, and generated code
(the list of includes changes depending on whether the {\em Simulation}
check-box is checked or not).

This support or library code should really be included into the project
using the scripting features of the tool, eg., using Quartus or Modelsim
Tcl commands, it should not use a Verilog-specific language feature
embedded within a generated source file. 

For more proof for this argument, consider that VHDL does not even
support this type of include construct.
%
\item
Because the SOPC System file \verb+sopc_system.v+ includes library
source directly, this source is compiled {\em every} time the SOPC
System is changed. This unnecessarily compiles source that has not
changed. Under Modelsim, the only source that would need to be
recompiled would be the interconnect and any new components added
to the system.
%
\item
The use of absolute paths in the \verb+include+ statements means that 
this code is not portable between machines, so the code should not be
checked into a code versioning system. Admittedly, this is generated 
code, so the use of the absolute paths could be tolerated.
%
\item The Verilog \verb+include+ statements are not used
consistently. In SOPC Builder, if you {\em do not} check the
{\em Simulation} checkbox, then the JTAG master components are not
listed in the \verb+include+ statements, whereas, if you do check
the check box, the components are listed. The JTAG master is a
{\em synthesizable} component, it should always be included!
If there are differences between synthesis and simulation,
then those differences should be hidden in the source code using
synthesis directives (which both Verilog and VHDL support).

In this particular example, the {\em synthesis} tool is provided the
path to the JTAG component in the \verb+sopc_builder.qip+ file.
This is a better solution, however, it is not reuseable by 
Modelsim.

\item Verilog and VHDL source files can be described via two
strings; the source file name and the library in which that
source should be compiled. The HDL source often has a compilation
order requirement, eg., VHDL packages need to be compiled before
use, and Verilog modules need to be compiled before they are
instantiated in other modules.

A general-purpose approach to including source would be for
SOPC Builder to generate a list of source files per library, or
a list of source file and library name pairs, with the list
in the appropriate compilation order. Tcl synthesis and
simulation scripts can then parse that list and issue the
appropriate Quartus or Modelsim commands to include that
source for compilation. This would allow source to be
included into projects using {\em scripting} features, not HDL
language features.
%
\end{itemize}

\item Pages~\pageref{bug:6a} and~\pageref{bug:6b}: The copying of library component source to
multiple project directories complicates verification.

The Modelsim simulator can be used to compile Verilog and VHDL source code
into libraries. The intention for those libraries is that they contain
the components that are reusable across multiple projects. For example,
Modelsim-ASE ships with the Altera LPM \verb+lpm+ and Megafunction 
\verb+altera_mf+ components pre-compiled for Verilog and VHDL in the 
directory \verb+c:/software/altera/11.1sp1/modelsim_ase/altera/+.
%
However, this is not how Quartus 11.1sp1 operates for SOPC Builder
library components. For example, in the SOPC Builder example
on page~\pageref{bug:6a},
%
\begin{itemize}
%
\item For synthesis, the JTAG-to-Avalon-MM master source code is 
copied from the Quartus II installation into the directory 
\verb+jtag_master+.
%
\item For simulation, the JTAG-to-Avalon-MM master source code is 
copied from the Quartus II installation into the directory 
\verb+jtag_master_sim+, and the \verb+sopc_system.v+ Verilog source,
\verb+test_bench+ \verb+include+ statements refer to this new 
directory, eg., \newline\verb+`include "jtag_master_sim/jtag_master.v"+.

In addition, the directory \verb+jtag_master_sim+, also contains
a {\em copy} of the \verb+jtag_master.v+ component instance
from the project directory (which is just one directory above
the copy). 
%
\end{itemize}
%
Why do all these copies complicate things? Well, if you are
verifying all of your designs using Modelsim, then you should
use Modelsim to build the library components into libraries,
and then reuse those libraries while verifying all of the
{\em project-specific} generated components. If there was
a valid argument for copying library components to the project
directory, then there should only be a {\em single} copy,
the same copy could be used for both synthesis and simulation,
and Modelsim could be used to create a project specific
SOPC builder component library. However, SOPC Builder copies
the library components a per-{\em component instance} basis.
For example, if you add another JTAG-to-Avalon-MM bridge 
component called \verb+jtag_master_two+, and regenerate the 
system with  {\em Simulation} checked, then two new directories
are created with additional copies of the library component source.

Quartus 10.1 would generate Verilog \verb+include+ statements
that referred to the original source files in the Quartus II
installation (the SOPC Builder IP directory). This method
at least points to the installed source area, but still uses
a Verilog-specific language feature to include source, whereas
scripting features should be used, as commented above.

The Qsys example on page~\pageref{bug:6a} also copies code,
but with a few differences relative to SOPC Builder;
%
\begin{itemize}
\item Qsys creates a copy of library source code in synthesis
and simulation directories;
%
\begin{verbatim}
qsys_system/synthesis/submodules
qsys_system/simulation/submodules
\end{verbatim}
%
\item The top-level \verb+qsys_system+ file no longer contains
Verilog \verb+include+ statements, the source files are instead
included in a tool-specific manner; Quartus is provided the
source information in the Tcl script
%
\begin{verbatim}
qsys_system/synthesis/qsys_system.qip
\end{verbatim}
%
while Modelsim is provided the files via the Tcl script
%
\begin{verbatim}
qsys_system\simulation\mentor\msim_setup.tcl
\end{verbatim}
%
The Modelsim Tcl script hard-codes the library mappings into
the relative directory \verb+./libraries+ making this script
difficult to support in a verification system that uses
multiple versions of Modelsim, eg., Modelsim-ASE and
Modelsim-SE (the binary library files produced by different
versions of Modelsim are not compatible). Page~\pageref{bug:9}
has comments on difficulties experienced with scripting
the Qsys design.

My preference would be to see these separate methods combined
into a parseable Tcl list containing the source file name and
the target library name. This would ease the writing of
custom synthesis and simulation scripts.

\end{itemize}

\item Page~\pageref{bug:7}: The Qsys system file does not contain all of the system settings.

The Qsys file \verb+.qsys+ does not preserve the state of the generate
tab checkboxes. The state of the checkboxes and the simulation pull-down
menus is stored in a preferences file. For example, in the Qsys 
example on page~\pageref{bug:7}, the Qsys {\em Generation} tab
options are stored in the XML file
\verb+.qsys_edit/preferences.xml+ in an XML entry called \verb+generation+,
containing key-value pairs for each of the GUI settings, eg.,
in the Qsys generation tab
%
\begin{itemize}
\item If the {\em Create block symbol file} checkbox is unchecked, that
selection is stored in the preferences file as 
\verb+<generation block_symbol_file="0"/>+. 
%
\item If the {\em Create simulation model} pull-down menu selects {\em Verilog},
that selection adds the key-value pair \verb+simulation="VERILOG"+
to the existing \verb+generation+ entry.
%
\end{itemize}
%
The use of an {\em additional} file to store the GUI settings
complicates the re-generation of a Qsys system with simulation 
support (from a minimal set of files), as now both the 
\verb+.qsys+ and preferences files are required. Without
the preferences files, the simulation files will {\em not} 
get generated.

This is a change from SOPC Builder, where the system (including
simulation settings) could be regenerated using only 
the \verb+.sopc+ file.

\item Page~\pageref{bug:8}: Port names exported from Qsys are {\em not} what
the user entered.

\end{enumerate}


\clearpage
% =================================================================
\section{Altera Documentation Web Links}
% =================================================================

\begin{itemize}
%
\item \href{http://www.altera.com/support/software/sof-quartus.html}
{Quartus II Software Support}
%
\item \href{http://www.altera.com/literature/lit-qts.jsp}
{Quartus II Development Software Documentation}
%
\item \href{http://www.altera.com/support/software/system/qsys/sof-qsys-index.html}
{Qsys System Integration Tool Support}
%
\item \href{http://www.altera.com/literature/lit-sop.jsp}{SOPC Builder Documentation}
%
\item \href{http://www.altera.com/support/software/system/sopc/sof-sopc_builder.html}{SOPC Builder Support}
%
\item \href{http://www.altera.com/literature/lit-nio2.jsp}
{NIOS II Processor Documentation}
%
\item \href{http://www.altera.com/support/ip/processors/nios2/ips-nios2_support.html}
{Nios II Embedded Design Suite Support}
%
\end{itemize}

%\clearpage
% -----------------------------------------------------------------
% Do the bibliography
% -----------------------------------------------------------------
%Note, you can't have spaces in the list of bibliography files
%
\bibliography{refs}
\bibliographystyle{plain}

% -----------------------------------------------------------------
\end{document}











